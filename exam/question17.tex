\documentclass[../main/document.tex]{subfiles}

\begin{document}
\section{Простые поля, расширения полей, поле разложения многочлена}
\begin{dfn}[Простое поле]
Поле - простое, если его подалгебры не являются полями
\end{dfn}
\begin{dfn}[Собственное подполе]
\end{dfn}
\begin{thm}
Любое просто поле изоморфно либо рациональным числам или полю вычетов по простому числу, то есть
$F$ - простое поле, тогда $F\simeq Q$ или $F\simeq \mathbb{Z}_p$, где $p\in\mathbb{Z}$ - простое
\begin{proof}
В поле есть $1$, поэтому можно строить кратные суммы единиц $(1+..+1)$. Строя такие суммы  мы или никогда не получим $0$ или получим
\begin{enumerate}
\item Никогда не получится $0$, то есть $k\cdot 1\neq 0\,(-(k\cdot 1)\neq 0)$ при $k>0$.

В поле для любого элемента есть обратный: $(k\cdot 1)^{-1}$ и $-(k\cdot 1)^{-1}$. В поле можно умножать: $(m\cdot 1)(k\cdot 1)^{-1}$. Так можно заметить что все элементы имеют вид
\begin{align*}
m\cdot 1&=(m\cdot 1)(1\cdot 1)^{-1}\\
k\cdot 1&=(1\cdot 1)(k\cdot 1)^{-1}
\end{align*}
Если $m\neq 0,k\neq 0$, то $(m\cdot 1)(k\cdot 1)^{-1}\neq 0$.
Так как $\{(m\cdot 1)(k\cdot 1)^{-1}\}$ образует поле и $F$ - простое, то $\{(m\cdot 1)(k\cdot 1)^{-1}\}$ образует всё поле.

Можно построить изоморфизм где $(m\cdot 1)(k\cdot 1)^{-1}\rightarrowtext{h}\frac{m}{k}$. Покажем что это так. Сначала нужно доказать что это гомоморфизм:

Да, это гомоморфизм

Так как поле - это кольцо, для $h$ существует $\Ker h$ и по \ref{kernel-is-ideal} $\Ker h$ - идеал. Так как поле - тело, то по \ref{only-trivial-ideal} существует только два идеала: $F$ и $\{0\}$. Ядро гомоморфизма является одним из этих идеалов, и так как оно не может быть равно всему полю $F$ оно равно $\{0\}$
Для того чтобы показать что $h$ - изоморфизм, нужно показать что это инъекция и сюръекция
\begin{enumerate}
\item Так как $\Ker h=\{0\}$ то по \ref{zero-kernel-injection} $h$ разнозначно
\item для каждого образа $\frac{m}{k}\in \mathbb{Q}$ есть прообраз $(m\cdot 1)(k\cdot 1)^{-1}\in F$
\end{enumerate}
Следовательно $F\simeq \mathbb{Q}$
\item $k\cdot 1=0$ для некоторого $k>0$

Выберем наименьшее $k>0$ для которого $k\cdot 1=0$. Мы можем получить элементы $0,1,2\cdot 1,3\cdot 1,...,(k-1)\cdot 1$. Докажем от противного что $k$ должно быть простым:

Так как $k$ не простое, то оно раскладывается $k=pq$, где $p,q>1,\,p,q<k$.
$$0=k\cdot 1=(p\cdot 1)(q\cdot 1)$$
поскольку $p,q<k$, то
$$(p\cdot 1)\neq 0\neq (q\cdot 1)$$
делители нуля. Противоречие, число не составное.

Возьмём $p=k$, $\mathbb{Z}_p=\{0,...,p-1\}$ - это кольцо (ассоциативное, коммутативное, с единицей), остаётся проверить наличие обратного. Пусть $x\neq 0$ и $x\in \mathbb{Z}_p$, тогда $\Nod(x,p)=1$. Из этого следует что $nx+mp=1$ для некоторых $n,m\in \mathbb{Z}_p$
\begin{align*}
nx+mp&=1\\
(nx+mp)\Mod{p}&=1\Mod{p}\\
nx\Mod{p}+mp\Mod{p}&=1\\
nx\Mod{p}&=1\\
n\Mod{p}\cdot x\Mod{p}&=1\\
n\Mod{p}\cdot x&=1
\end{align*}
$n\Mod{p}$ - обратный для произвольного $x$, соответственно $\mathbb{Z}_p$ - поле.
\end{enumerate}
\end{proof}
\end{thm}
\begin{cnsq}
Внутри каждого поля есть простое подполе
\begin{proof}

\end{proof}
\end{cnsq}
\begin{dfn}[Характеристика поля]
Для некоторого поля $F$ его характеристика это
\begin{enumerate}
\item если $k\cdot 1\neq0$ для всех $k>0$, то $0$ - характеристика поля $F$
\item если $k\cdot 1=0$ для некоторого $k>0$, то $k$ - характеристика поля $F$ ($F$ - поле конечной характеристики)
\end{enumerate}
\end{dfn}
\begin{dfn}[Неразложимый многочлен]
Неразложимый многочлен - многочлен, который не раскладывается на множители, ни один из которых не является многочленом нулевой степени.
\end{dfn}
\begin{exm}[Пример неразложимого многочлена]

\end{exm}
\begin{cnsq}
\begin{enumerate}
\item Многочлен 1 степени всегда неразложим
\item Многочлен 2 или 3 степени неразложим $\Leftrightarrow$ не имеет корней
\item Если многочлен степени большей 3 не разложим, то он не имеет корней
\end{enumerate}
\begin{proof}

\end{proof}
\end{cnsq}
\begin{cnsq}\label{irreducible-polynomial}
Неразложимый многочлены - простые элементы кольца многочленов
\begin{proof}

\end{proof}
\end{cnsq}
\textcolor{red}{
\begin{thm}\label{prime-generates-prime}
$R$ - кольцо главных идеалов, $c$ - простой элемент, тогда $cR$ - простой идеал
\begin{proof}
Пусть $c$ - простой элемент, допустим что $cR$ - не простой идеал, тогда найдутся $a,b\not\in cR$ такие что $ab\in cR$. Сумма идеалов $dR=aR+cR$ - тоже идеал. Потом я не пойму ПОЧЕМУ.
\end{proof}
\end{thm}
\begin{thm}
$R$ - кольцо главных идеалов, $I$ - простой идеал, тогда $I$ - максимальный идеал
\begin{proof}
Пусть дан простой идеал $I=cR$, дальше магия
\end{proof}
\end{thm}}
\begin{cnsq}
Если $p$ - неразложимый многочлен, тогда порождённый им идеал является максимальным
\begin{proof}
Следует из двух предыдущих и \ref{irreducible-polynomial} или из \ref{prime-generates-maximal} и \ref{irreducible-polynomial}
\end{proof}
\end{cnsq}

\textcolor{red}{
\begin{cnsq}
$\bigslant{F[x]}{\langle p\rangle}$ - поле
\begin{proof}
Следует из того что факторкольцо по протому элементу - это поле, но здесь такой теоремы (пока) нет
\end{proof}
\end{cnsq}}
\begin{thm}
Для каждого многочлена существует расширение поля, в котором он разложится на линейные множители.
\begin{proof}
Пусть $p(x)\in P[x]$. Индукция по степени многочлена $p$:

Базис. $\deg p=1$. $p(x)=ax+b$ - линейный, то есть уже разложен

Индукционный шаг. Предположим $p$ раскладывается на два многочлена $p=q\cdot s$, тогда по индукционному предположению для этих многочленов существует поле где они разложатся. 

Теперь предположим что $p$ не раскладывается. Построим $\bigslant{F[y]}{\langle p(y)\rangle}=F'$ - расширение $F$. Это будет расширением потому что можно построить изоморфизм $h:F\to F',\,h(y)=y+\langle p(y)\rangle$

Пусть $\alpha = y+\langle p(y)\rangle$ - корень многочлена $p$ в $F'$, тогда
\begin{align*}
p(y+\langle p(y)\rangle)&=\sum\limits_{i=0}^np_i(y+\langle p(y)\rangle)^i\\
&=\sum\limits_{i=0}^np_i(y^i+\langle p(y)\rangle)\\
&=(\sum\limits_{i=0}^np_iy^i)+\langle p(y)\rangle\\
&=p(y)+\langle p(y)\rangle\\
&=0
\end{align*}
действительно, $\alpha = y+\langle p(y)\rangle$ - корень многочлена $p$ в $F'$
\end{proof}
\end{thm}
\begin{exm}[Пример расширения поля]

\end{exm}
\textcolor{red}{
\begin{cnsq}
Если $F$ - конечное поле, то поле расширений многочлена $p$ тоже конечно
\begin{proof}
По индукции
\end{proof}
\end{cnsq}}
\begin{cnsq}
Пусть $p\in P[x]$ и $\deg p =n$, тогда количество корней $p$ с учётом кратности будет $\leq n$ и существует поле где оно равно $n$
\begin{proof}
Пусть $F'$ - расширение $F$, в над которым многочлен раскладывается на линейные множители. Тогда $p(x)=a_0(x-a_1)^{n_1}...(x-a_k)^{n_k}$. Если $\deg p =n$, то $n_1+...n_k=n$
\end{proof}
\end{cnsq}

\end{document}