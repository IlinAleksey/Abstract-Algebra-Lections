\documentclass[../main/document.tex]{subfiles}

\begin{document}
\section{Простые поля, расширения полей, поле разложения многочлена}
\begin{dfn}[Простое поле]
Поле -простое, если оно не содержит собственных подполей
\end{dfn}
\begin{dfn}[Собственное подполе]
\end{dfn}
\begin{thm}
$F$ - простое поле, тогда $F\simeq Q$ или $F\simeq \mathcal{Z}_p$
\begin{proof}

\end{proof}
\end{thm}
\begin{cnsq}
Внутри каждого поля есть простое подполе
\begin{proof}

\end{proof}
\end{cnsq}
\begin{dfn}[Характеристика поля]

\end{dfn}
\begin{dfn}[Неразложимый многочлен]
Неразложимый многочлен - многочлен, который не раскладывается на множители
\end{dfn}
\begin{cnsq}
\begin{enumerate}
\item Многочлен 1 степени всегда неразложим
\item Многочлен 2 или 3 степени неразложим $\Leftrightarrow$ не имеет корней
\item Если многочлен степени большей 3 не разложим, то он не имеет корней
\end{enumerate}
\end{cnsq}
\begin{cnsq}
Неразложимый многочлены - простые элементы кольца многочленов
\end{cnsq}
\begin{thm}
$R$ - кольцо главных идеалов, $c$ - простой элемент, тогда $cR$ - простой идеал
\end{thm}
\begin{cnsq}
Если $p$ - неразложимый многочлен, тогда порождёныый им мдеал является максимальным
\end{cnsq}
\begin{cnsq}
$\bigslant{F(x)}{\langle p\rangle}$ - поле
\end{cnsq}
\begin{thm}
Для каждого многочлена существует расширение поля, в котором он разложится на линейные множители.
\begin{proof}

\end{proof}
\end{thm}
\begin{cnsq}
Если $F$ - конечное поле, то поле расширений многочлена $p$ тоже конечно
\end{cnsq}
\begin{cnsq}
$deg p =n$
\begin{proof}

\end{proof}
\end{cnsq}
\end{document}