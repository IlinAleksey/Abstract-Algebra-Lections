\documentclass[../main/document.tex]{subfiles}

\begin{document}
\section{Простые поля, расширения полей, поле разложения многочлена}
\begin{dfn}[Простое поле]
Поле - простое, если его подалгебры не являются полями
\end{dfn}
\begin{dfn}[Собственное подполе]
\end{dfn}
\begin{thm}
Любое просто поле изоморфно либо рациональным числам или полю вычетов по простому числу, то есть
$F$ - простое поле, тогда $F\simeq Q$ или $F\simeq \mathbb{Z}_p$, где $p\in\mathbb{Z}$ - простое
\begin{proof}
В поле есть $1$, поэтому можно строить кратные суммы единиц $(1+..+1)$. Строя такие суммы  мы или никогда не получим $0$ или получим
\begin{enumerate}
\item Никогда не получится $0$, то есть $k\cdot 1\neq 0\,(-(k\cdot 1)\neq 0)$ при $k>0$.

В поле для любого элемента есть обратный: $(k\cdot 1)^{-1}$ и $-(k\cdot 1)^{-1}$. В поле можно умножать: $(m\cdot 1)(k\cdot 1)^{-1}$. Так можно заметить что все элементы имеют вид
\begin{align*}
m\cdot 1&=(m\cdot 1)(1\cdot 1)^{-1}\\
k\cdot 1&=(1\cdot 1)(k\cdot 1)^{-1}
\end{align*}
Если $m\neq 0,k\neq 0$, то $(m\cdot 1)(k\cdot 1)^{-1}\neq 0$.
Так как $\{(m\cdot 1)(k\cdot 1)^{-1}\}$ образует поле и $F$ - простое, то $\{(m\cdot 1)(k\cdot 1)^{-1}\}$ образует всё поле.

Можно построить изоморфизм где $(m\cdot 1)(k\cdot 1)^{-1}\rightarrowtext{h}\frac{m}{k}$. Покажем что это так. Сначала нужно доказать что это гомоморфизм:

Да, это гомоморфизм

Так как поле - это кольцо, для $h$ существует $\Ker h$ и по \ref{kernel-is-ideal} $\Ker h$ - идеал. Так как поле - тело, то по \ref{only-trivial-ideal} существует только два идеала: $F$ и $\{0\}$. Ядро гомоморфизма является одним из этих идеалов, и так как оно не может быть равно всему полю $F$ оно равно $\{0\}$
Для того чтобы показать что $h$ - изоморфизм, нужно показать что это инъекция и сюръекция
\begin{enumerate}
\item Так как $\Ker h=\{0\}$ то по \ref{zero-kernel-injection} $h$ разнозначно
\item для каждого образа $\frac{m}{k}\in \mathbb{Q}$ есть прообраз $(m\cdot 1)(k\cdot 1)^{-1}\in F$
\end{enumerate}
Следовательно $F\simeq \mathbb{Q}$
\item $k\cdot 1=0$ для некоторого $k>0$

Выберем наименьшее $k>0$ для которого $k\cdot 1=0$. Мы можем получить элементы $0,1,2\cdot 1,3\cdot 1,...,(k-1)\cdot 1$. Докажем от противного что $k$ должно быть простым:

Так как $k$ не простое, то оно раскладывается $k=pq$, где $p,q>1,\,p,q<k$.
$$0=k\cdot 1=(p\cdot 1)(q\cdot 1)$$
поскольку $p,q<k$, то
$$(p\cdot 1)\neq 0\neq (q\cdot 1)$$
делители нуля. Противоречие, число не составное.

Возьмём $p=k$, $\mathbb{Z}_p=\{0,...,p-1\}$ - это кольцо (ассоциативное, коммутативное, с единицей), остаётся проверить наличие обратного
\end{enumerate}
\end{proof}
\end{thm}
\begin{cnsq}
Внутри каждого поля есть простое подполе
\begin{proof}

\end{proof}
\end{cnsq}
\begin{dfn}[Характеристика поля]

\end{dfn}
\begin{dfn}[Неразложимый многочлен]
Неразложимый многочлен - многочлен, который не раскладывается на множители
\end{dfn}
\begin{cnsq}
\begin{enumerate}
\item Многочлен 1 степени всегда неразложим
\item Многочлен 2 или 3 степени неразложим $\Leftrightarrow$ не имеет корней
\item Если многочлен степени большей 3 не разложим, то он не имеет корней
\end{enumerate}
\end{cnsq}
\begin{cnsq}
Неразложимый многочлены - простые элементы кольца многочленов
\end{cnsq}
\begin{thm}
$R$ - кольцо главных идеалов, $c$ - простой элемент, тогда $cR$ - простой идеал
\end{thm}
\begin{cnsq}
Если $p$ - неразложимый многочлен, тогда порождёныый им мдеал является максимальным
\end{cnsq}
\begin{cnsq}
$\bigslant{F(x)}{\langle p\rangle}$ - поле
\end{cnsq}
\begin{thm}
Для каждого многочлена существует расширение поля, в котором он разложится на линейные множители.
\begin{proof}

\end{proof}
\end{thm}
\begin{cnsq}
Если $F$ - конечное поле, то поле расширений многочлена $p$ тоже конечно
\end{cnsq}
\begin{cnsq}
$deg p =n$
\begin{proof}

\end{proof}
\end{cnsq}

\end{document}