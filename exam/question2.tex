\documentclass[../main/document.tex]{subfiles}

\begin{document}

\section{Подалгебры, порождающие элементы,\\ вложения}

\begin{dfn}[Подалгебра]
Подалгебра - алгебра $\mathcal{B}=(B,J)$ является подалдгеброй $\mathcal{A}=(A,I)$, если $B\subseteq A$ и $J(f)$ - ограничение $I(f)$ на $B$ для всякого $f$
\end{dfn}
\begin{dfn}[Ограничение операции]
Ограничение операции - $n$-местная операция $g$ на $B$ является ограничением операции $f$ множеством $B$ если 
$$g(b_1,...,b_n)=f(b_1,...,b_n)$$
для любых $b_1,...,b_n$ из $B$
\end{dfn}
\begin{exm}[Пример ограничения операции]

\end{exm}
\begin{exm}[Пример подалгебры]
Пример подалгебры:
$$(\mathbb{C},+,\cdot)\supseteq (\mathbb{R},+,\cdot)\supseteq (\mathbb{Q},+,\cdot)$$
\begin{proof}

\end{proof}
\end{exm}
\begin{cnsq}
Отношение "является подалгеброй" транзитивно
$$A\subseteq B, B\subseteq C \Rightarrow A\subseteq C$$
\begin{proof}

\end{proof}
\end{cnsq}
\begin{thm}\label{carrier-theorem}
Если $\mathcal{A}=(A,I)$ - алгебра, то $B$ $(B\subseteq A; B\neq \emptyset)$ является носителем некоторой подалгебры тогда и только тогда, когда $B$ замкнута относительно сигнатурной операции в алгебре $\mathcal{A}$
\begin{proof}
\begin{enumerate}
\item $\Rightarrow$

$B$ - носитель некоторой подалгебры $\mathcal{B}=(B,J)$ и $B\subseteq A$, тогда
$$f^{\mathcal{A}}(b_1,...,b_n)=f^{\mathcal{B}}(b_1,...,b_n)\in B$$
$B$ замкнута относительно сигнатурной операции в алгебре $\mathcal{A}$
\item $\Leftarrow$
$B$ замкнута относительно сигнатурной операции в алгебре $\mathcal{A}$, тогда

$J(f)$ - функция на $B$

$J(f)(b_1,...,b_n)=f^{\mathcal{A}}(b_1,...,b_n)\in B$

$J(f)$ - ограниение $f^{\mathcal{A}}$  на $B$

следовательно $\mathcal{B}=(B,J)$ - подалгебра и $B$ - её носитель
\end{enumerate}
\end{proof}
\end{thm}
\begin{exm}[Пример на \ref{carrier-theorem}]

\end{exm}
\begin{thm}
Пусть $\mathcal{A}=(A,I)$ - алгебра, $\mathcal{B}_k$ - подалгебры, такие что $\bigcap\limits_k \mathcal{B}_k\neq \emptyset$, тогда $\bigcap\limits_k \mathcal{B}_k$ - носитель подалгебры
\begin{proof}
Пусть $f^{(n)}\in\Sigma$, $b_1,...,b_n\in \bigcap\limits_k \mathcal{B}_k$, тогда
\begin{align*}
\Rightarrow&\text{ по определению пересечения}&b_1,...,b_n&\in\mathcal{B}& \text{для всех }k\\
\Rightarrow&\text{ по }\ref{carrier-theorem}&f^{\mathcal{A}}(b_1,...,b_n)&\in\mathcal{B}& \text{для всех }k\\
\Rightarrow&\text{ по определению пересечения}&f^{\mathcal{A}}(b_1,...,b_n)&\in\bigcap\limits_k \mathcal{B}_k& 
\end{align*}
\end{proof}
\end{thm}
\begin{dfn}[Порождённая подалгебра]
Пусть $\mathcal{A}=(A,I),\,x\subseteq A,\,X\neq\emptyset,\,\mathcal{B}_k$ - всевозможные подалгебры, включающие $X$, тогда $\bigcap \mathcal{B}_k$ - подалгебра, порождённвя $X$.
\end{dfn}
\begin{thm}
$\mathcal{A}$ - алгебра, $X\subseteq A,\,X\neq\emptyset,\,\mathcal{B}$ - подалгебра, порождённвя $X$ тогда и только тогда, когда $\mathcal{B}$ состоит из всевозможных $t^{\mathcal{A}}(x_1,...,x_n)$ для $x_1,...,x_n\in X$
\begin{proof}
Достаточность($\Leftarrow$). Пусть $\mathcal{B}$ состоит из всевозможных $t^{\mathcal{A}}(x_1,...,x_n)$ для $x_1,...,x_n\in X$. Пусть $B_k\in \mathcal{A}$ - подалгебры такие что $X\subseteq B_k$, тогда
\begin{align*}
x_1,...,x_n&\in B_k\\
t^{\mathcal{A}}(x_1,...,x_n)&\in B_k\\
t^{\mathcal{A}}(x_1,...,x_n)&\in \bigcap\limits_X B_k\\
t^{\mathcal{A}}(x_1,...,x_n)&\in \mathcal{B}
\end{align*}

Необходимость($\Rightarrow$). Предположим, что найдётся $b\in \mathcal{B}$, что $b\neq t^{\mathcal{A}}(x_1,...,x_n)$ для любых $t$ и $x_1,...,x_n\in X$, 
Пусть $C=\{t^{\mathcal{A}}(x_1,...,x_n): t - \text{терм, }x_1,...,x_n\in X\}$, следовательно $b\not\in C$, $x_i\in C$ и $X\subseteq C$. 

$C$ является подалгеброй: пусть $c_1,...,c_m\in C$ и
\begin{align*}
c_1&=t_1^{\mathcal{A}}(x_1,...,x_n)
&\vdots
c_m&=t_m^{\mathcal{A}}(x_1,...,x_n)
\end{align*}
$$f^{\mathcal{A}}(c_1,...,c_m)=f^{\mathcal{A}}(t_1^{\mathcal{A}}(x_1,...,x_n),...,t_m^{\mathcal{A}}(x_1,...,x_n))$$
По определению терма $f^{\mathcal{A}}(t_1^{\mathcal{A}}(x_1,...,x_n),...,t_m^{\mathcal{A}}(x_1,...,x_n))$ тоже является термом, содержащий переменные $x_1,...,x_n$, следовательно $C$ замкнуто по сигнатурной операции $A$ и по \ref{carrier-theorem} является подалгеброй.

$C=\mathcal{B}_k$ ждя некоторого $k$, $\mathcal{B}=\bigcap\limits_k \mathcal{B}_k$. Так как $b\not\in C$, то $b\not\in \mathcal{B}$, что является противоречием.
\end{proof}
\end{thm}
\begin{dfn}[Разнозначное отображение]
$f$ - разнозначное, если $f(x)\neq f(y)$ при $x\neq y$
\end{dfn}
\begin{dfn}[Вложение]
$h:\mathcal{A}\rightarrow \mathcal{B}$ - вложение $\mathcal{A}$ в $\mathcal{B}$, если $h$ - разнозначное отображение и
$$h(f^{\mathcal{A}}(a_1,...,a_n))=f^{\mathcal{B}}(h(a_1),...,h(a_n))$$
говорят "$\mathcal{A}$ вкладывается в $\mathcal{B}$"
\end{dfn}
\begin{thm}
$h:\mathcal{A}\rightarrow \mathcal{B}$ - вложение $\mathcal{A}$ в $\mathcal{B}$, тогда
\begin{enumerate}
\item образ $h$ - $\mathcal{C}$, подалгебра в $\mathcal{B}$
\item $h:\mathcal{A}\simeq\mathcal{C}$
\end{enumerate}
\begin{proof}
\begin{enumerate}
\item Пусть $c_1,...,c_n\in \rng h$, тогда $c_1=h(a_1),...,c_n=h(a_n)$ и
$$h(f^{\mathcal{A}}(a_1,...,a_n))=f^{\mathcal{B}}(h(a_1),...,h(a_n))=f^{\mathcal{B}}(c_1,...,c_n)\in \rng h$$
Элементы образа $h$ замкнуты относительно сигнатурных операций $\mathcal{B}$
\textcolor{red}{\item $\mathcal{C}=\rng h$,
$h:A\leftrightarrow C$,
$\Rightarrow h$ - изоморфизм}
\end{enumerate}

\end{proof}
\end{thm}
\end{document}