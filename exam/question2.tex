\documentclass[../main/document.tex]{subfiles}

\begin{document}

\section{Подалгебры, порождающие элементы,\\ вложения}

\begin{dfn}[Подалгебра]
Подалгебра - алгебра $\mathcal{B}=(B,J)$ является подалдгеброй $\mathcal{A}=(A,I)$, если $B\subseteq A$ и $J(f)$ - ограничение на $B$ для всякого $f$
\end{dfn}
\begin{dfn}[Ограничение операции]
Ограничение операции - $n$-местная операция $g$ на $B$ является ограничением операции $f$ множеством $B$ если 
$$g(b_1,...,b_n)=f(b_1,...,b_n)$$
для любых $b_1,...,b_n$ из $B$
\end{dfn}
\begin{exm}[Пример ограничения операции]

\end{exm}
\begin{exm}[Пример подалгебры]
Пример подалгебры:
$$(\mathbb{C},+,\cdot)\supseteq (\mathbb{R},+,\cdot)\supseteq (\mathbb{Q},+,\cdot)$$
\begin{proof}

\end{proof}
\end{exm}
\begin{cnsq}
Отношение "является подалгеброй" транзитивно
$$A\subseteq B, B\subseteq C \Rightarrow A\subseteq C$$
\begin{proof}

\end{proof}
\end{cnsq}
\begin{thm}\label{carrier-theorem}
Если $\mathcal{A}=(A,I)$ - алгебра, то $B$ $(B\subseteq A; B\neq \emptyset)$ является носителем некоторой подалгебры тогда и только тогда, когда $B$ замкнута относительно сигнатурной операции в алгебре $\mathcal{A}$
\begin{proof}
\begin{enumerate}
\item $\Rightarrow$

$B$ - носитель подалгебры $\mathcal{B}=(B,J)$ и $B\subseteq A$, тогда
$$f^{\mathcal{A}}(b_1,...,b_n)=f^{\mathcal{B}}(b_1,...,b_n)\in B$$
$B$ замкнута относительно сигнатурной операции в алгебре $\mathcal{A}$
\item $\Leftarrow$
$B$ замкнута относительно сигнатурной операции в алгебре $\mathcal{A}$, тогда

$J(f)$ - функция на $B$

$J(f)(b_1,...,b_n)=f^{\mathcal{A}}(b_1,...,b_n)\in B$

$J(f)$ - ограниение $f^{\mathcal{A}}$  на $B$

следовательно $\mathcal{B}=(B,J)$ - подалгебра и $B$ - её носитель
\end{enumerate}
\end{proof}
\end{thm}
\begin{exm}[Пример на \ref{carrier-theorem}]

\end{exm}

\end{document}