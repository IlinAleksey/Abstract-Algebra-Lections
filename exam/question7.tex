\documentclass[../main/document.tex]{subfiles}

\begin{document}
\section{Группы, абелевы группы, циклические\\ группы. Вложение моноида в группу}
\begin{dfn}[Группа]
Группа - моноид, в котором все элементы обратимы
\end{dfn}
\begin{dfn}[Тривиальная группа]
Тривиальная группа - группа, состоящая из одного элемента
\end{dfn}
\begin{thm}
Если $M$ - моноид и $G\subseteq M$ - подмножество обратимых элементов, то $G$ - группа
\begin{proof}
$G\subseteq M$ следовательно $G$ ассоциативна, 
$e$ - обратимый следовательно $G$ имеет нейтральный элемент. 
Надо доказать замкнутость: $x*y\in G$

$x',y'$ - обратные к $x$ и $y$ элементы, тогда
$$(x*y)*(y'*x')=x*(y*y')*x'=x*e*x'=x*x'=e$$
$$(y'*x')*(x*y)=y'*(x'*x)*y=y*e*y'=y*y'=e$$
$x*y$ обратим $\Rightarrow xy\in G$

если $x\in G$, то $x'*x=x*x'=e$, тогда $x'$  имеет обратный элемент, тогда $x'\in G$. Любой элемент $G$ имеет обратный.

$G$ - группа. Теорема доказана.

\end{proof}
\end{thm}
\begin{dfn}[Абелева группа]
Абелева группа - группа, в которой $xy=yx$
\end{dfn}
\begin{dfn}[Циклическая группа]
Циклическая группа - группа, порождённая одним элементом. $<a>$ - циклическая группа порождённая $a$.

$(\omega,+,0)$ изоморфно бесконечной циклической группе

моноид типа $(i,j)$ изоморфен конечной циклической группе
\end{dfn}

\begin{thm}
$\mathcal{G}=\langle a\rangle$, тогда $\mathcal{G}\cong (\mathbb{Z},+)$ или  $\mathcal{G}\cong (\mathbb{Z}_n,+)$ для некоторого $n$
\begin{proof}
Пусть $\mathcal{M}$ - подмоноид, порождённый $a$. $\mathcal{M}$ - циклический
\begin{enumerate}
\item $\mathcal{M} \morphtext{h} (\omega,+,0)$

$x\in \mathcal{M}$ $x^{-1}$ $xx^{-1}=e$

$x\in \mathcal{M}$ $x\neq e$ $x^{-1}\neq \mathcal{M}$

$$h(x)+h(x^{-1})=h(xx^{-1})=h(e)=0$$

$h(x)=h(x^{-1})=0$

$h(x)=h(x^{-1})=e$

$x\in \mathcal{M}$

$h(x^{-1})=-h(x)$

\item $\mathcal{M}$ - конечный $(i,j)$ моноид, если $i>0$, то в $\mathcal{M}$ есть нееденичный идемпотент, следовательно он необратимый, следовательно в группе должно быть $i=0$
$$a^xa^y=\begin{cases} a^{x+y} &,\mbox{если } x+y<j\\
a^{(x+y)\pmod j}&,\mbox{если } x+y\geq j\end{cases}$$
$\mathcal{M}$ - группа
$$a^x=a^{j-x}=a^{j\pmod j}=e$$
$\mathcal{M}$ - группа порождённая $a$, $\mathcal{M}=\mathcal{G}$

$h:a^x\rightarrow x$
\end{enumerate}
\end{proof}
\end{thm}
\begin{thm}
В циклической группе существуют нетривиальные группы тогда и только тогда когда она бесконечна или $n$ в $(\mathbb{Z}_n,+)$ составное 
\begin{proof}
\begin{enumerate}
\item $\Rightarrow$ пусть имеется $(\mathbb{Z}_n,+)$, $n$ - простое, $a\neq 0$, $a<n$, $a$ и $n$ взаимно простые, следовательно $xa+yn=1$. пусть $b\in \mathbb{Z}$, тогда 
$$b=b\cdot 1=b(ax+yn)=(bx)a+(by)n$$
$$(\underbrace{a+a+...+a}_{bx})\mod n=(b-(by)n)\mod n=b\mod n=b$$

Таким образом любые подгруппы, содержащие не только $0$ содержат $\mathbb{Z}_n$
\item $\Leftarrow$ 
\begin{enumerate}
    \item бесконечная циклическая группа имеет нетривиальную подгруппу
    \item пусть $n=xy$, тогда $(\mathbb{Z}_{xy},+)\supseteq \{0,x,2x,...,(y-1)x\}$
  \end{enumerate}
\end{enumerate}
\end{proof}
\end{thm}
\begin{thm}[Теорема Гротендика]
Каждый коммутативный моноид, в котором все элементы сократимы можно вложить в группу
\begin{proof}
Пусть $M$ - коммутативный моноид, $G'=M\times M=(a,b)$, где $a,b\in M$, $(a_1,b_1)(a_2,b_2)=(a_1a_2,b_1b_2)$, $(e_1,e_2)$ - нейтральный элемент.

Пусть $(a,b)\equiv (c,d)\Leftrightarrow ad=bc$. Является ли $\equiv$ конгруэнтностью?

\begin{enumerate}
\item $(a,b)\equiv(a,b)$, $ab=ba$
\item $(a,b)\equiv(c,d)$, $ad=bc\Rightarrow cb=da\Rightarrow (c,d)\equiv (a,b)$ 
\item $(a,b)\equiv(c,d)\equiv(u,v)\Rightarrow(a,b)\equiv(u,v)$
\end{enumerate}

Надо доказать:
$$(a_1,b_1)\equiv(a_2,b_2), (c_1,d_1)\equiv(c_2,d_2)\Rightarrow(a_1c_1,b_1d_1)\equiv(a_2c_2,b_2d_2)$$

\begin{multline*}
(a_1,b_1)\equiv(a_2,b_2), (c_1,d_1)\equiv(c_2,d_2)\Rightarrow \\ 
a_1b_2=b_1a_2, c_1d_2=d_1c_2\Rightarrow a_1b_2c_1d_2=b_1a_2d_1c_2\Rightarrow \\
 (a_1c_1)(b_2d_2)=(b_1d_1)(a_2c_2)\Rightarrow \\
 (a_1c_1,b_1d_1)\equiv(a_2c_2,b_2d_2)
\end{multline*}

$(a,b)\equiv (c,d)\Leftrightarrow ad=bc$ - конгруэнтность
\vspace{1em}

Пусть $G = \bigslant{G'}{\equiv}$ надо доказать что $G$ - группа и $M$ вкладывается в $G$

$$ab=ba\Rightarrow abe=ab=ba=bae\Rightarrow (ab,ba)\equiv(e,e)$$
$$\widehat{(a,b)}*\widehat{(b,a)}=\widehat{(ab,ba)}=\widehat{(e,e)}$$
$\Rightarrow$ каждый элемент $G$ имеет обратный $\Rightarrow$ $G$ - группа

Пусть $h:M\rightarrow G$ и $h(a)=\widehat{(a,e)}$, тогда
$$h(ab)=\widehat{(ab,e)}=\widehat{(a,e)}\widehat{(b,e)}=h(a)h(b)$$
$$h(e)=\widehat{(e,e)}$$
$h$ - гомоморфизм

Пусть $h(a)=h(b)$
$$\widehat{(a,e)}=\widehat{(b,e)}\Rightarrow (a,e)\equiv(b,e)\Rightarrow ae=eb \Rightarrow a=b$$
следовательно $h$ - инъекция, следовательно $h$ - вложение

\end{proof}
\end{thm}
\begin{exm}[Пример на теорему Гротендика]
\end{exm}
\end{document}