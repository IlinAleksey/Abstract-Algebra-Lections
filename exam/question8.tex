\documentclass[../main/document.tex]{subfiles}

\begin{document}
\section{Группы пере становок, задание групп определяющими соотношениями.}
\begin{dfn}[Группа перестановок]
Группа перестановок - группа перестановок множества $S$ называется группа всех биекций $f:S\rightarrow S$. $(F,\circ,e,{ }^{-1})$
\end{dfn}
\begin{exm}[Пример группы перестановок]
\end{exm}
\begin{dfn}[Симметрическая группа порядка]
Симметрическая группа порядка $n$: $S$ - конечно и состоит из $n$ элементов. $(A,\circ,e,{ }^{-1})$, $A$ - множество автоморфизмов $h:S\rightarrow S$
\end{dfn}
\begin{exm}[Пример симметрической группы] \label{TriangleGroup}
Пример симметрической группы:

\begin{tikzpicture}
\draw (0,0) -- (2,3.46410161514) -- (4,0)-- (0,0);
\draw	(0,0) node[anchor=north east] {2}
		(2,3.46410161514) node[anchor=south] {1}
		(4,0) node[anchor=north west] {3};
\end{tikzpicture}

$A=\{e,r_1,r_2,s_1,s_2,s_3\}$
\begin{itemize}

  \item $e$ - тождественное преобразование
  \item $r_1, r_2$ - поворот на $120^{\circ}$ и $240^{\circ}$ соответственно
  \item $s_1, s_2, s_3$ - оборот вокруг высоты, идущей из первой, второй и третьей вершины соответственно
\end{itemize}

$$\mathbf{D}_3=(A,\circ)$$

\begin{table}[H]
\centering
\caption*{Таблица умножения $\circ$}
\renewcommand*{\arraystretch}{1.4}
\begin{tabular}{c|c|c|c|c|c|c}
  & $e$ & $r_1$ & $r_2$ & $s_1$& $s_2$ & $s_3$  \\ \hline
$e$ & $e$ & $x$ & $e$ & $x$& $e$ & $x$ \\ \hline
$r_1$ & $e$ & $x$ & $e$ & $x$& $e$ & $x$ \\ \hline
$r_2$ & $e$ & $x$ & $e$ & $x$& $e$ & $x$ \\ \hline
$s_1$ & $e$ & $x$ & $e$ & $x$& $e$ & $x$ \\ \hline
$s_2$ & $e$ & $x$ & $e$ & $x$& $e$ & $x$ \\ \hline
$s_3$ & $x$ & $x$& $e$ & $x$& $e$ & $x$ \\ 
\end{tabular}
\end{table}
\end{exm}

\begin{exm}[задание групп определяющими соотношениями]

\end{exm}
\end{document}