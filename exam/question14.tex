\documentclass[../main/document.tex]{subfiles}

\begin{document}
\section{Гомоморфизмы колец, идеалы,\\ фактор-кольца}
\begin{dfn}[Гомоморфизм колец]
$h:R\to S$ - гомоморфизм, определённый так: $a\equiv b\Leftrightarrow h(a)=h(b)$
\end{dfn}
\begin{dfn}[Ядро кольца]\label{kernel}
$h:R\to S$ - гомоморфизм, тогда ядро кольца $\Ker h=\{a\in R:h(a)=0\}$ 
\end{dfn}
\begin{thm}\label{kernel-subring}
Ядро кольца - подкольцо
\begin{proof}
Пусть $\Ker h$ - ядро кольца $R$ по гомоморфизму $R\to S$, тогда
\begin{enumerate}
\item $\Ker h\neq \emptyset$
\item $\forall x,y\in \Ker h:h(x+(-y))=h(x)+h(-y)\eqtext{\ref{group-homomorphism-properties}}
h(x)-h(y)\eqtext{\ref{kernel}}0\Rightarrow x+(-y)\in \Ker h$
\item $\forall x,y\in \Ker h:h(x\circ y)=h(x)\circ h(y)=0\circ 0=0\Rightarrow  x\circ y\in \Ker h$
\end{enumerate}
По \ref{subring-test} ядро $\Ker h$ является группой
\end{proof}
\end{thm}

\begin{dfn}[Идеал]\label{ideal}
$R$ - кольцо, $\mathcal{I}\subseteq R$ - идеал (левый, правый, двусторонний), если
\begin{enumerate}
\item $\mathcal{I}$ - подкольцо
\item для любого $x\in R$ $x\mathcal{I}\subseteq\mathcal{I}$ (левый идеал), $\mathcal{I} x\subseteq\mathcal{I}$ (правый идеал)
\end{enumerate}
\end{dfn}

\begin{thm}\label{kernel-is-ideal}
Ядро кольца - идеал
\begin{proof}
Пусть $\Ker h$ - ядро кольца $R$ по гомоморфизму $R\to S$, тогда
\begin{enumerate}
\item по теореме \ref{kernel-subring}
\item
\begin{enumerate}
\item $\forall x\in R, y\in \Ker h: h(xy)=h(x)h(y)=h(x)*0=0\Rightarrow xy\in \Ker h\Rightarrow x\Ker h\subseteq \Ker h$
\item $\forall x\in R, y\in \Ker h: h(yx)=h(y)h(x)=0*h(x)=0\Rightarrow yx\in \Ker h\Rightarrow \Ker h*x\subseteq \Ker h$
\end{enumerate}

\end{enumerate}
По определению идеала ядро $\Ker h$ является идеалом
\end{proof}
\end{thm}
\begin{exm}[Пример идеалов]

\begin{thm}\label{zero-kernel-injection}
Пусть $R,S$ - кольца, $h:R\to S$ - гомоморфизм. Если $\Ker h=\{0\}$, то $h$ - вложение
\begin{proof}
Пусть $\Ker h=\{0\},\,x,y\in A$. Пусть $h(x)=h(y)=b$, тогда
\begin{align*}
 &          & h(x)-h(y)&=b-b & &\\
 &          & &=0         &    &\\
\Rightarrow & & h(x-y)&=0 & & \ref{homomorphism}\\
\Rightarrow & & (x-y)&\in\Ker h & & \ref{kernel}\\
\Rightarrow & & x-y&=0 h & & \\
\Rightarrow & & x&=y & & 
\end{align*}
Так как $x,y$ были произвольными, то $h$ - вложение
\end{proof}
\end{thm}

\end{exm}
\begin{lemma}\label{unit-trivial}
Если $R$ - кольцо, $a\neq 0$, $a\in R$ и $1\in aR$, то $aR=R$
\begin{proof}
Так как $1\in aR$, то $a$ обратим, то есть существует $a^{-1}\in R$,следовательно
$$aR\supseteq aa^{-1}R= R$$
Так как $R\subseteq aR$ и $aR\subseteq R$, то $aR=R$
\end{proof}
\end{lemma}
\begin{thm}\label{only-trivial-ideal}
$R$ - ассоциативное кольцо с единицей или $R$ - тело или $R$ тогда и только тогда когда в $R$ Нет других идеалов, кроме $\{0\}$ и $R$
\begin{proof}
Так как $R$ - ассоциативное кольцо с единицей или или тело, то для каждого $a$ существует обратное $a^{-1}$. По лемме \ref{unit-trivial} для всех $a\neq 0$ $aR=R$. Остаётся только $a=0$, который образует идеал $\{0\}$
\end{proof}
\end{thm}
\begin{dfn}[Булевое кольцо]

\end{dfn}
\begin{thm}
Пусть $I$ - двухсторонний идеал в $R$, тогда отношение $\equiv: x\equiv y \Leftrightarrow x-y\in I$ является конгруэнтностью
\begin{proof}

\end{proof}
\end{thm}
\begin{cnsq}
Существует фактор-алгебра $\bigslant{R}{\equiv}$, такая что ???
\end{cnsq}

\begin{cnsq}
$I=\Ker h$, где $h:R\to\bigslant{R}{\equiv} $
\begin{proof}

\end{proof}
\end{cnsq}
\begin{dfn}[Простой идеал]
Пусть $R$ - ассоциативное, коммутативное кольцо с единицей, тогда $I$ - простой идеал, если $ab\in I\Leftrightarrow a\in I$ или $b\in I$
\end{dfn}
\begin{dfn}[Максимальный идеал]
Пусть $R$ - ассоциативное, коммутативное кольцо с единицей, тогда $I$ - максимальный идеал, если для любого идеала $J: I\subseteq J, I\neq J$ выполняется $J=R$
\end{dfn}
\begin{dfn}[Главный идеал]
Пусть $R$ - ассоциативное, коммутативное кольцо с единицей, тогда $I$ - главный идеал, если для некоторого $a\in R$ $I=aR$
\end{dfn}
\begin{exm}[??????]

\end{exm}
\begin{lemma}
Если $I$ и $J$ - идеалы, то $I+J$ тоже идеал
\begin{proof}

\end{proof}
\end{lemma}
\begin{thm}
Пусть $R$ - ассоциативное, коммутативное кольцо с единицей, $I$ - идеал, тогда
\begin{enumerate}
\item $I$ - простой идеал $\Leftrightarrow$ $\bigslant{R}{I}$ - целостное
\item $I$ - максимальный идеал $\Leftrightarrow$ $\bigslant{R}{I}$ - поле
\end{enumerate}
\begin{proof}

\end{proof}
\end{thm}
\end{document}