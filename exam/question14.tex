\documentclass[../main/document.tex]{subfiles}

\begin{document}
\section{Гомоморфизмы колец, идеалы, фактор-кольца}
\begin{dfn}[Гомоморфизм колец]
$h:R\to S$ - гомоморфизм, определённый так: $a\equiv b\Leftrightarrow h(a)=h(b)$
\end{dfn}
\begin{dfn}[Ядро кольца]
$h:R\to S$ - гомоморфизм, тогда ядро кольца $\Ker h=\{a\in R:h(a)=0\}$ 
\end{dfn}
\begin{thm}
Ядро кольца - подкольцо
\end{thm}
\begin{dfn}[Идеал]
$R$ - кольцо, $\mathcal{I}\subseteq R$ - идеал (левый, правый, двусторонний), если
\begin{enumerate}
\item $\mathcal{I}$ - подкольцо
\item для любого $x\in R$ $x\mathcal{I}\subseteq\mathcal{I}$ (левый идеал), $\mathcal{I} x\subseteq\mathcal{I}$ (правый идеал)
\end{enumerate}
\end{dfn}
\begin{exm}[Пример идеалов]

\end{exm}
\begin{thm}
$R$ - ассоциативное кольцо с единицей или $R$ - тело или $R$ тогда и только тогда когда в $R$ Нет других идеалов, кроме $\{0\}$ и $R$
\end{thm}
\begin{dfn}[Булевое кольцо]

\end{dfn}
\begin{thm}
Пусть $I$ - двухсторонний идеал в $R$, тогда отношение $\equiv: x\equiv y \Leftrightarrow x-y\in I$ является конгруэнтностью
\begin{proof}

\end{proof}
\end{thm}
\begin{cnsq}
Существует фактор-алгебра $\bigslant{R}{\equiv}$, такая что ???
\end{cnsq}

\begin{cnsq}
$I=\Ker h$, где $h:R\to\bigslant{R}{\equiv} $
\begin{proof}

\end{proof}
\end{cnsq}
\begin{dfn}[Простой идеал]
Пусть $R$ - ассоциативное, коммутативное кольцо с единицей, тогда $I$ - простой идеал, если $ab\in I\Leftrightarrow a\in I$ или $b\in I$
\end{dfn}
\begin{dfn}[Максимальный идеал]
Пусть $R$ - ассоциативное, коммутативное кольцо с единицей, тогда $I$ - максимальный идеал, если для любого идеала $J: I\subseteq J, I\neq J$ выполняется $J=R$
\end{dfn}
\begin{dfn}[Главный идеал]
Пусть $R$ - ассоциативное, коммутативное кольцо с единицей, тогда $I$ - главный идеал, если для некоторого $a\in R$ $I=aR$
\end{dfn}
\begin{exm}[??????]

\end{exm}
\begin{lemma}
Если $I$ и $J$ - идеалы, то $I+J$ тоже идеал
\begin{proof}

\end{proof}
\end{lemma}
\begin{thm}
Пусть $R$ - ассоциативное, коммутативное кольцо с единицей, $I$ - идеал, тогда
\begin{enumerate}
\item $I$ - простой идеал $\Leftrightarrow$ $\bigslant{R}{I}$ - целостное
\item $I$ - максимальный идеал $\Leftrightarrow$ $\bigslant{R}{I}$ - поле
\end{enumerate}
\begin{proof}

\end{proof}
\end{thm}
\end{document}