\documentclass[../main/document.tex]{subfiles}

\begin{document}
\section{Циклические моноиды, свободные моноиды .}
\begin{dfn}[Свободный моноид]
Свободный моноид - моноид, элементами которого являются конечные последовательности (строки) элементов носителя моноида. Свободный моноид на множестве $A\neq \emptyset$ это $\mathcal{A}=(A^*;\&,\varepsilon)$, $A^*$ - множество всех слов в алфавите $A$, $\&$ - конкатенация, $\varepsilon$ - пустое слово.
\end{dfn}
\begin{thm}
Любой моноид, порождённый элементами множества, на котором есть свободный моноид, является гомоморфным образом этого моноида
\begin{proof}
Пусть $A\neq \emptyset$, $\mathcal{A}=(A^*;\&)$,\\ $\mathcal{B}=(\left\{t^{\mathcal{B}}(a_1,...,a_n): a_1,...,a_n\in A\right\};*)$ и $h:\mathcal{A}\rightarrow \mathcal{B}$ - Гомоморфизм
$$h(a_1...a_n)=(a_1,...,a_n)^{\mathcal{B}}$$
$$h(\varepsilon)=e^{\mathcal{B}}$$
Надо доказать свойство гомоморфизма:
$$h(u\&v)=h(u)*h(v)$$
Пусть $u=a_1...a_n$, $v={a'}_1...{a'}_n$, тогда
$$h(u\&v)=h(uv)=h(a_1...a_n{a'}_1...{a'}_n)=(a_1...a_n{a'}_1...{a'}_n)^{\mathcal{B}}$$
\begin{multline*}
h(u)*h(v)=h(a_1...a_n)*h({a'}_1...{a'}_n)=\\
(a_1...a_n)^{\mathcal{B}}*({a'}_1...{a'}_n)^{\mathcal{B}}=(a_1...a_n{a'}_1...{a'}_n)^{\mathcal{B}}
\end{multline*}
Из этого следует что $h(u\&v)=h(u)*h(v)$
\end{proof}
\end{thm}
\begin{exm}[Примеры свободных моноидов и их гомоморфных образов]

Пусть дан алфавит $A=\{1\}$, который образует множество слов $A^*=\{\varepsilon,1,11,...\}$ и моноид $\mathcal{A}=(A^*;\&,\varepsilon)$, тогда
\begin{enumerate}
\item $\mathcal{B}=({1};\cdot,1)$, порождённый элементами $A$ является гомоморфным образом $\mathcal{A}$, $h:A\rightarrow B$, $h(1...1)=1$
\item $\mathcal{C}=(\omega;+,0)$, порождённый элементами $A$(натуральные числа можно получить сложением единицы) является гомоморфным образом $\mathcal{A}$, $h:A\rightarrow B$, $h(\underbrace{1...1}_n)=n$
\end{enumerate}
\end{exm}

\begin{dfn}[Циклический моноид]
Циклический моноид - моноид порождённый одним элементом. $<a>$ - циклический моноид, порождённый элементом $a$. 

$e,a,a^1,a^2,a^3,...$ - элементы моноида $<a>$

\begin{enumerate}
\item $a^i\neq a^j$ при $i\neq j$

$h:<a>\rightarrow ({\left\{a\right\}}^*;\&)$, $h(a^i)=i$ - изоморфизм.
\item $a^i=a^j$ при $i\neq j$
$$k=i+(k-i)=i+y(j-i)+r$$
$$r=(k-i)mod(j-i)$$
$$r<j-i$$
тогда
\begin{multline*}
a^k=a^i\underbrace{a^{j-i}...a^{j-i}}_{y}a^r=\\
(a^ia^{j-i})\underbrace{a^{j-i}...a^{j-i}}_{y-1}a^r\stackrel{(a^ia^{j-i}=a^{i+j-i}=a^j=a^i)}{=}a^i\underbrace{a^{j-i}...a^{j-i}}_{y-1}a^r=\\
a^ia^r=a^{i+r} (r<j-i; i+r<j)
\end{multline*}
\end{enumerate}
к чему весь этот список?
\end{dfn}

\begin{exm} [Пример циклического моноида]\label{cyclical-exm}
$<a>=(\{e,a,...\};*)$

Таблица умножения $(*)$ -
\begin{table}[h]
\centering
\renewcommand*{\arraystretch}{1.4}
\begin{tabular}{|l|l|l|l|}
\hline
  & $e$ & $a$ & $a^2$ \\ \hline
$e$ & $a$ & $a$ & $a^2$ \\ \hline
$a$ & $a$ & $a^2$ & $a$ \\ \hline
$a^2$ & $a^2$ & $a$ & $a^2$ \\
\hline
\end{tabular}
\end{table}
\end{exm}
\begin{thm}
Если $j$ - наименьшее число такое что $a^i=a^j$ для какого-то $i<j$, то $<a>$ содержит ровно $j$ элементов
\begin{proof}
$$\underbrace{e,a^1,...,a^{j-1}}_\text{нет равных},\underbrace{a^j=a^i,a^{j+1}=a^{i+1},...}_\text{повоторяющиеся}$$
если $j$ - номер наименьшего повтора, тогда
$$a^x*a^y=
\begin{cases}
	a^{x+y},& \text{если } x+y<j\\
	a^{i+(x+y-i)mod(j-i)},& \text{если } x+y\geq i
\end{cases}
$$
\begin{align*}
x+y&=k, & k&=i+(k-i\cdot z+r\\
& & r&=(k-i)mod(j-i)\\
& & a^k&=a^{i+z}
\end{align*}
$$a^{x+y}=a^k=a^{i+(x+y-i)mod(j-i)}$$

\end{proof}
\end{thm}
\textcolor{red}{
\begin{dfn}[Моноид типа $(i,j-i)$]
Моноид типа $(i,j-i)$ - моноид с элементами $$???$$
\end{dfn}}

\textcolor{red}{
\begin{thm}
В моноиде типа $(i,j-i)$, где $i>0$ существует идемпотент $b\neq e$
\begin{proof}
???
\end{proof}
\end{thm}}
\end{document}