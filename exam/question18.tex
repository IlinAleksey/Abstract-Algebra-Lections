\documentclass[../main/document.tex]{subfiles}

\begin{document}
\section{Конечные поля}

\begin{dfn}[Конечное поле]

\end{dfn}

\textcolor{red}{
\begin{cnsq}
Конечные поля имеют конечную характеристику
\begin{proof}
\begin{align*}
\underbrace{1+...+1}_n&=\underbrace{1+...+1}_m&  n&=m\\
\underbrace{1+...+1}_n&=0& & &
\end{align*}
Что это вообще такое
\end{proof}
\end{cnsq}}

\begin{thm}
Если $F$ - конечное поле характеристики $p$, то $\abs{F}=p^k$
\begin{proof}
Так как $F$ - конечное поле $\mathbb{Z}_p\subseteq F$, тогда $F$ - линейное пространство (\textcolor{red}{почему это линейное пространство}) над $\mathbb{Z}_p$, в таком случае имеется базис $e_1,...,e_k$. Пусть $a\in F$б тогда
$$a=a_1e_1+...+a_ke_k\quad a_1,...,a_k\in\mathbb{Z}_p$$
И так как $\abs{\mathbb{Z}_p}=p$, то $\abs{F}=p^k$ - количество комбинаций $a_1,...,a_k$
\end{proof}
\end{thm}

\textcolor{red}{
\begin{cnsq}
Если $m\neq p$, то поля из $m$ элементов не существует
\begin{proof}
???
\end{proof}
\end{cnsq}}

\begin{thm}[Мечта школьника]
Если $F$ - поле характеристики $p$, то
$$(x+y)^p=x^p+y^p$$
\begin{proof}
Пусть $x,y\in F$, тогда по формуле бинома ньютона
$$(x+y)^p=\sum\limits_{i=0}^pC^i_px^iy^{p-i}$$
Рассмотрим первый и последний элемент этой суммы. По формуле сочетания
$$C^0_p=\frac{p!}{0!(p-0)!}=1$$

$$C^p_p=\frac{p!}{p!(p-p)!}=1$$
поэтому 
\begin{align*}
(x+y)^p&=\sum\limits_{i=0}^pC^i_px^iy^{p-i}\\
&=C^0_px^0y^{p}+\sum\limits_{i=1}^{p-1}C^i_px^iy^{p-i}+C^p_px^py^{0}\\
&=y^p+\sum\limits_{i=1}^{p-1}C^i_px^iy^{p-i}+x^p
\end{align*}
Рассмотрим оставшуюся часть суммы, то есть для $i\neq 0\neq p$. По формуле сочетания
$$C^i_p=\frac{p!}{i!(p-i)!}=p\cdot c_i\quad i\neq 0\neq p$$
где $c_i$ - некоторое число, зависимое от $i$.
Подставляя $C^i_p$ получаем
$$
\sum\limits_{i=1}^{p-1}C^i_px^iy^{p-i}=
\sum\limits_{i=1}^{p-1}pc_ix^iy^{p-i}=
\sum\limits_{i=1}^{p-1}\underbrace{(1+...+1)}_p c_ix^iy^{p-i}=0
$$
так как элемент $p\in F$ равен нулю.
Таким образом
$$(x+y)^p=y^p+\sum\limits_{i=1}^{p-1}C^i_px^iy^{p-i}+x^p=x^p+y^p$$
\end{proof}
\end{thm}
\begin{thm}
Если $F$ - поле характеристики $p$, то
$$({(x+y)^p})^k=({x^p})^k+({y^p})^k$$
\begin{proof}
\begin{align*}
(x+y)^{p^k}&=((x+y)^p)^{p^{k-1}}\\
&=(x^p+y^p)^{p^{k-1}}\\
&=((x^p+y^p)^p)^{p^{k-2}}\\
&=...\\
&=(x^{p^{k-1}}+y^{p^{k-1}})^{p^1}\\
&=x^{p^k}+y^{p^k}
\end{align*}
\end{proof}
\end{thm}
\begin{thm}
Если $F$ - конечное поле и $\abs{F}=m$, тогда существует корень уравнения типа $x^{m-1}-x$
\begin{proof}
Пусть $F'=\{F\setminus \{0\},\cdot,1,-1\}$, $F'$ является группой и $\abs{F'}=m-1$. Пусть $a\in F'$, тогда по \ref{power-order-of-group} 
$$a^{m-1}=1$$
То есть все ненулевые элементы группы удовлетворяют $x^{m-1}=0$.

Так как $x^m-x=x(x^{m-1}-1)$, то и нулевой элемент и ненулевые элементы являются корнями этого уравнения.
\end{proof}
\end{thm}
\begin{thm}
Если существует поле $F$, такое что $\abs{F}=p^k$, то существует поле $F'$, такое что $\abs{F'}=p^{k'}$, при любом $k'\leq k$
\begin{proof}
Если $a|b$, то $(x^a-1)|(x^b-1)$ (\textcolor{red}{как так}). Предположим $b=ac$, то есть
$$x^b=(x^a)^c-1=(x^a-1)((x^a)^{c-1}+(x^a)^{c-2}+...+x^a+1)$$
$F$ - корни многочлена $x^{p^k}-x=x(x^{p^{k-1}}-1)$ 

\textcolor{red}{Что дальше в этой теореме?}
\end{proof}
\end{thm}
\textcolor{red}{Что дальше в этой главе?}
\end{document}