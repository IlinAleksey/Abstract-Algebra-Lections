\documentclass[../main/document.tex]{subfiles}

\begin{document}
\section{Конечные поля}

\begin{dfn}[Конечное поле]

\end{dfn}

\begin{cnsq}
Конечные поля имеют конечную характеристику
\end{cnsq}

\begin{thm}
Если $F$ - конечное поле характеристики $p$, то $\abs{F}=p^k$
\begin{proof}

\end{proof}
\end{thm}
\begin{cnsq}
Если $m\neq p$, ТО поля из $m$ элементов не существует
\end{cnsq}
\begin{thm}
Если $F$ - поле характеристики $p$, то
$$(x+y)^p=x^p+y^p$$
\begin{proof}

\end{proof}
\end{thm}
\begin{thm}
Если $F$ - поле характеристики $p$, то
$$({(x+y)^p})^k=({x^p})^k+({y^p})^k$$
\begin{proof}

\end{proof}
\end{thm}
\begin{thm}
Если $F$ - конечное поле и $\abs{F}=m$, тогда существует корень уравнения типа $x^m-1$
\begin{proof}
Мамка твоя
\end{proof}
\end{thm}
\end{document}