\documentclass[../main/document.tex]{subfiles}

\begin{document}
\section{Евклидовы кольца, кольца главных идеалов, факториальные кольца}
\begin{dfn}[Евклидово кольцо]\label{euclidean-ring}
$R$ - ассоциативное, коммутативное кольцо с единицей, $R$ - евклидово, если для каждого элемента $a$ этого кольца существует его норма $\Vert a\Vert$.
\end{dfn}
\begin{dfn}[Евклидова норма]\label{euclidean-norm}
Это некоторая функция элемента кольца, такая что
\begin{enumerate}
\item $\Vert a\Vert \in \omega$
\item если $a,b\neq 0$, то $\Vert ab\Vert \geq \max(\Vert a\Vert,\Vert b\Vert)$
\item если $a\neq 0$, то для любого $b$ существуют $d$ и $r$ такие что $b=da+r$ и $\Vert r\Vert < \Vert a\Vert$ или $r=0$
\end{enumerate}
\end{dfn}
\begin{dfn}[Кольцо главных идеалов]
Кольцо главных идеалов - кольцо, в котором все идеалы главные
\end{dfn}
\begin{thm}
Каждое евклидово кольцо - кольцо главных идеалов
\begin{proof}

\end{proof}
\end{thm}
\begin{thm}\label{ascending-chain-condition}
В кольце главных идеалов $R$ не существует бесконечно возрастающей цепи идеалов
$$I_0\subseteq I_1\subseteq I_2\subseteq ...$$
\begin{proof}
Пусть $I_0\subseteq I_1\subseteq I_2\subseteq ...$ - возрастающая цепь идеалов и $I=\cup_{i=0}^\infty I_i$, докажем что $I$ - идеал
\begin{enumerate}
\item докажем что $I$ - подкольцо по теореме \ref{subring-test}
\begin{enumerate}
\item $I$ замкнут по сложению и умножению, покажем на элементах $a,b\in I$. В таком случае в цепи есть идеалы $I_j$ и $I_k$, такие что $a\in I_j$ и $b\in I_k$. Если $m\geq \max(j,k)$ то оба элемента $a$ и $b$ принадлежат $I_m$, поэтому принадлежат и $a+b$ и $ab$. Поэтому $a+b\in I$ и $ab\in I$
\item $0\in I$ потому что $0\in I_i$ для всякого $i$
\item Пусть $a\in I$. Тогда $a\in I_j$ Для какого-то $j$, в этом случае $-a\in I_j$, следовательно $-a\in I$
\end{enumerate}
следовательно $I$ - подкольцо
\item Пусть $a\in I$. Тогда $a\in I_j$ Для какого-то $j$. Пусть $r$ - любой элемент $R$, тогда $ra\in I_j$, следовательно $ra\in I$. Следовательно $rI\subseteq I$
\end{enumerate}
по определению \ref{ideal} $I$ - идеал.

Так как $R$ - КГИ и $I$ - идеал, то существует $a\in R$, такое что $I=aR$. Так как $a\in I$ существует $n$ такой что $a\in I_n$. Следовательно $aR\subseteq I_n$. По определению $I$ $I_n\subset I=aR$. $I_n$ и $I$ входят друг в друга следовательно $I=I_n$. Если брать любое $m\geq n$ то должно выполнятся условие $I\subseteq I_m$. Это возможно только если $I_m=I$.

 Следовательно после некоторого конечного элемента $n$ цепь идеалов перестаёт возрастать
\end{proof}
\end{thm}
\begin{dfn}[Простой элемент]
Пусть $R$ - ассоциативное, коммутативное кольцо с единицей, тогда $a$ - простой, если из $a=bc$ следует что $b$ или $c$ обратимы
\end{dfn}
\begin{dfn}[Факториальное кольцо]\label{factorial-ring}
Пусть $R$ - ассоциативное, коммутативное кольцо с единицей, тогда $R$ - факториальное кольцо, если для каждого элемента $a\in R$
\begin{enumerate}
\item существует простые $b_1,...,b_n$, такие что $a=b_1...b_n$
\item если $a=c_1...c_m$, где $c_1,...,c_m$ - простые, то $m=n$, существует перестановка $\sigma$, Такая что $c_i=e_ib_{\sigma(i)}$ Для обратимого $e_i$
\end{enumerate} 
\end{dfn}
\begin{thm}
Существует нефакториальное кольцо
\end{thm}
\begin{lemma}
Если $R$ - кольцо, $a\in R$ и $1\in aR$, то $aR=R$
\begin{proof}
Так как $1\in aR$, то $a$ обратим, то есть существует $a^{-1}\in R$,следовательно
$$aR\supseteq aa^{-1}R= R$$
Так как $R\subseteq aR$ и $aR\subseteq R$, то $aR=R$
\end{proof}
\end{lemma}
\begin{thm}
$R$ - целостное кольцо и $a\neq 0$, Тогда следующие условия эквивалентны
\begin{enumerate}
\item $a$ - необратимый
\item $aR\neq R$
\item Для любого $b\neq 0$ $abr\neq bR$
\item для некоторого $b\neq 0$ $abr\neq bR$
\end{enumerate}
\begin{proof}

$1\Rightarrow 2$

$ab\neq 1$ для любого $b$, соответствено $aR\not\ni 1$, следовательно $aR\neq R$

$2\Rightarrow 3$

Пусть $b\neq 0$. Допустим $abR=br\ni b$. Пусть для некоторого $r\in R$ верно $abr=b$, следовательно
$$arb-b=0\Rightarrow (ar-1)b=0\Rightarrow ar-1=0\Rightarrow ar=1$$
то есть $1\in aR$, следовательно $aR=R$, Противоречие.

$3\Rightarrow 4$

Если для любого  $b\neq 0$ верно $abr\neq bR$, то верно и для некоторого

$4\Rightarrow 1$

Допустим $a$ - обратимый, то есть существует $r\in R$, такой что $ar=1$, получается
$$abR=baR\subseteq bR$$
и
$$bR=1\cdot bR=arbR=abrR\subseteq abR$$
следовательно $bR=abR$, что противоречит 4, следовательно $a$ необратим
\end{proof}
\end{thm}
\begin{thm}\label{proper-factorization}
Если $R$ - КГИ, то каждый необратимый элемент отличный от нуля раскладывается в конечное произведение простых элементов
\begin{proof}
Пусть $a\in R$, $a\neq 0$, и $a$ - необратимый
\begin{enumerate}
\item Сначала покажем что $a$ имеет в разложении простой множитель. Если $a$ простой, то разложение завершено. Если нет, то $a=a_1b_1$, где ни $a_1$ ни $b_1$ необратимые. Тогда $a\in a_1R$ и $aR \subset a_1R$. Включение строгое, потому что если $aR=a_1R$, то для некоторого $r\in R$ было бы $a_1=ar$ и $a=arb_1$. Так как $R$ - целостное и $rb_1=1$, то $b_1$ - обратимый, что противоречит разложению $a=a_1b_1$, где ни $a_1$ ни $b_1$ необратимые.

Если $a_1$ не простой, то можно сказать $a_1=a_2b_2$, где ни $a_2$ ни $b_2$ необратимые. Получается
$$aR \subset a_1R \subset a_2R$$
где каждое включение строгое. Если $a_2$ не простое то можно продолжить цепь, но по теореме \ref{ascending-chain-condition} цепь нельзя продолжать бесконечно  и после конечного числа шагов она закончится идеалом $a_rR$, где $a_r$ - простое число. Следовательно в разложении $a$ есть некоторый простой элемент $a_r$
\item Теперь покажем что $a$ раскладывается в произведение простых элементов $R$. Если $a$ не простое, то по пункту 1 можно сказать $a=p_1c_1$, где $p_1$ - простое число и $c_1$ необратимое. Поэтому $aR$ строго вкладывается в $c_1R$. Если $c_1$ не простой, то $c_1=p_2c_2$ где $p_2$ - простое число и $c_2$ необратимое. Можно построить строго возрастающую цепь идеалов
$$aR \subset c_1R \subset c_2R$$
Эта цепь должна остановиться после конечного числа шагов на идеале $c_rR$, где $c_r$ - простой. Тогда
$$a=p_1p_2...p_rc_r$$
разложение на конечное число простых множителей
\end{enumerate}
\end{proof}
\end{thm}
\begin{lemma}
Пусть $I$ - идеал КГИ $R$. Тогда $I$ является максимальным тогда и только тогда когда $I=pR$, где $p$ - простой
\begin{proof}

\end{proof}
\end{lemma}

\begin{thm}
пусть $R$ - целостное кольцо главных идеалов, тогда $R$ - факториальное
\begin{proof}
Для того чтобы показать что $R$  - факториальное, надо показать что оно удовлетворяет условиям из \ref{factorial-ring}:
\begin{enumerate}
\item по теореме \ref{proper-factorization}
\item Надо показать что если $a=c_1...c_m=b_1,...,b_n$, где $c_1,...,c_m,b_1,...,b_n$ - простые, то $m=n$, существует перестановка $\sigma$, Такая что $c_i=e_ib_{\sigma(i)}$ Для обратимого $e_i$

Предположим что $n\geq m$. Так как $c_1|a$, то $c_1|b_1,...,b_n$, то есть $c_1|b_j$ для какого-то $j$. Можно переставить местами так что $c_1|b_1$. Тогда $b_1=c_1e_1$ для какого-то обратимого $e_1\in R$. Следовательно
$$c_1c_2...c_m=e_1c_1b_2...b_n$$
и
$$c_2...c_m=e_1b_2...b_n$$
Продолжая процесс получается
$$1=e_1e_1...e_mb_{m+1}b_n$$
Так как ни один из $b_i$ необратим, получается $m=n$ и $c_i=e_ib_{\sigma(i)}$. Покажем что существует такая $\sigma:\{1,...,m\}\to\{1,...,m\}$ что $\sigma$ - биекция. Определим $\sigma(i)=$ минимальный $j$, такой что $b_j|c_i$ и $j\not\in\{\sigma(1),...,\sigma(i-1)\}$. Нужно доказать что такой $j$ всегда найдётся, что $\sigma$ инъективна и сюръективна.
 
%$b_j|\frac{c_1...c_n}{b_{\sigma(1)}...b_{\sigma(i-1)}}$


\end{enumerate}
\end{proof}
\end{thm}
\end{document}