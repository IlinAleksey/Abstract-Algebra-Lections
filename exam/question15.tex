\documentclass[../main/document.tex]{subfiles}

\begin{document}
\section{Евклидовы кольца, кольца главных идеалов, факториальные кольца}
\begin{dfn}[Евклидово кольцо]
$R$ - ассоциативное, коммутативное кольцо с единицей, $R$ - евклидово, если для каждого элемента $a$ этого кольца существует его норма $\Vert a\Vert$.
\end{dfn}
\begin{dfn}[Евклидова норма]
Это некоторая функция элемента кольца, такая что
\begin{enumerate}
\item $\Vert a\Vert \in \omega$
\item если $a,b\neq 0$, то $\Vert ab\Vert \geq \max(\Vert a\Vert,\Vert b\Vert)$
\item если $a\neq 0$, то для любого $b$ существуют $d$ и $r$ такие что $b=da+r$ и $\Vert r\Vert < \Vert a\Vert$ или $r=0$
\end{enumerate}
\end{dfn}
\begin{dfn}[Кольцо главных идеалов]
Кольцо главных идеалов - кольцо, в котором все идеалы главные
\end{dfn}
\begin{thm}
Каждое евклидово кольцо - кольцо главных идеалов
\begin{proof}

\end{proof}
\end{thm}
\begin{thm}
В кольце главных идеалов не существует бесконечно возрастающей цепи идеалов
$$I_0\subseteq I_1\subseteq I_2\subseteq ...$$
\begin{proof}

\end{proof}
\end{thm}
\begin{dfn}[Простой элемент]
Пусть $R$ - ассоциативное, коммутативное кольцо с единицей, тогда $a$ - простой, если из $a=bc$ следует что $b$ или $c$ обратимы
\end{dfn}
\begin{dfn}[Факториальное кольцо]
Пусть $R$ - ассоциативное, коммутативное кольцо с единицей, тогда $R$ - факториальное кольцо, если для каждого элемента $a\in R$
\begin{enumerate}
\item существует простые $b_1,...,b_n$, такие что $a=b_1...b_n$
\item если $a=$
\end{enumerate} 
\end{dfn}
\begin{thm}
$R$ - целостное кольцо и $a\neq 0$, Тогда следующие условия эквивалентны
\begin{enumerate}
\item $a$ - необратимый
\item $aR\neq R$
\item Для любого $b\neq 0$ $abr\neq bR$
\item для некоторого $b\neq 0$ $abr\neq bR$
\end{enumerate}
\begin{proof}

\end{proof}
\end{thm}
\begin{thm}
пусть $R$ - целостное кольцо главных идеалов, тогда $R$ - факториальное
\begin{proof}

\end{proof}
\end{thm}
\end{document}