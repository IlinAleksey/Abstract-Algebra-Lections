\documentclass[../main/document.tex]{subfiles}

\begin{document}
\section{Евклидовы кольца, кольца главных идеалов, факториальные кольца}
\begin{dfn}[Евклидово кольцо]
$R$ - ассоциативное, коммутативное кольцо с единицей, $R$ - евклидово, если для каждого элемента $a$ этого кольца существует его норма $\Vert a\Vert$.
\end{dfn}
\begin{dfn}[Евклидова норма]
Это некоторая функция элемента кольца, такая что
\begin{enumerate}
\item $\Vert a\Vert \in \omega$
\item если $a,b\neq 0$, то $\Vert ab\Vert \geq \max(\Vert a\Vert,\Vert b\Vert)$
\item если $a\neq 0$, то для любого $b$ существуют $d$ и $r$ такие что $b=da+r$ и $\Vert r\Vert < \Vert a\Vert$
\end{enumerate}
\end{dfn}
\end{document}