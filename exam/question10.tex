\documentclass[../main/document.tex]{subfiles}

\begin{document}
\section{Гомоморфизмы групп, нормальные подгруппы, фактор-группа}
\begin{dfn}[Нормальная подгруппа]
Нормальная подгруппа - подгруппа, у которой любой левый смежный класс совпадает с правым
\end{dfn}

\begin{exm}
Пример нормальных групп
$$\langle r_1\rangle=\{r_0,r_1,r_2\}\subseteq \mathcal{D}_3$$
$$r_i\langle r_1\rangle=r_i\{r_0,r_1,r_2\}=\{r_{0+i},r_{1+i},r_{2+i}\}=\langle r_1\rangle$$
$$\langle r_1\rangle r_i=\{r_0,r_1,r_2\}r_i=\{r_{0+i},r_{1+i},r_{2+i}\}=\langle r_1\rangle$$
$$r_i\langle r_1\rangle=\langle r_1\rangle r_i$$
$$s_i\langle r_1\rangle=\{s_ir_0,s_ir_1,s_ir_2\}=\{s_i,s_{i-1},s_{i+1}\}$$
$$\langle r_1\rangle s_i=\{r_0s_i,r_1s_i,r_2s_i\}=\{s_i,s_{i+1},s_{i-1}\}$$
$$s_i\langle r_1\rangle=\langle r_1\rangle s_i$$
$\langle r_1\rangle$ - нормальная подгруппа

\end{exm}

\begin{thm}
Если $\mathcal{G}$ - группа, $\mathcal{H}\subseteq \mathcal{G}$, и $\equiv$ - отношение принадлежности к одному левому смежному классу, то $\equiv$ - отношение эквивалентности
\begin{proof}
\begin{enumerate}
\item Рефлексивность $a\in a\mathcal{H}\Rightarrow a\equiv a$
\item Симметричность $a\equiv b\Rightarrow a \in x\mathcal{H}, b\in x\mathcal{H}\Rightarrow b\equiv a$
\item Транзитивность $a\equiv b$, $b\equiv c\Rightarrow $
\begin{align*}
a,b&\in x\mathcal{H} & a &=xh_a & b=x&h_b \\
b,c&\in y\mathcal{H} & b &=yh'_b & c=y&h_c 
\end{align*} 
$$xh_b=yh'_b\Rightarrow x=yh'_bh^{-1}_b\Rightarrow a=y\underbrace{h'_bh^{-1}_bh_a}_{\mathcal{H}}$$
$$
\begin{rcases}
c \in y\mathcal{H}\\
a\in y\mathcal{H}
\end{rcases}
a\equiv c
$$
\end{enumerate}
\end{proof}
\end{thm}
\begin{dfn}[Факторгруппа]
Рассмотрим группу $G$ и ее нормальную подгруппу $H$. Пусть $G/H$ — множество смежных классов $G$ по $H$. Определим в $G/H$ операцию умножения по следующему правилу: $aH\cdot bH=(ab)H$
\end{dfn}

\begin{thm}
Определение произведения смежных классов корректно. То есть произведение смежных классов не зависит от выбранных представителей $a$ и $b$
\begin{proof}
Пусть $aH,bH\in G/H,\,a_1=a\cdot h_a\in aH,\,b_1=b\cdot h_b\in bH$. Докажем, что $abH=a_1 b_1 H$. Достаточно показать, что $a_1\cdot b_1 \in abH$.

В самом деле, $a_1\cdot b_1=a\cdot h_a\cdot b\cdot h_b=a\cdot b\cdot (b^{-1}\cdot h_a\cdot b)\cdot h_b$. Элемент $h = (b^{-1}\cdot h_a\cdot b)$ лежит в $H$ по свойству нормальности $H$. Следовательно, $a\cdot b\cdot h\cdot h_b\in abH$.
\end{proof}
\end{thm}

\begin{dfn}[Гомоморфизм групп]
Если $G$ и $H$ - группа, $h:G\rightarrow H$ и $h(a*b)=h(a)*h(b)$, то $h$ - гомоморфизм
\end{dfn}
\begin{cnsq}\label{group-homomorphism-properties}
Гомоморфизм групп обладает следующими свойствами:
\begin{enumerate}
\item $h(e)=e$
\item $h(a^{-1})={h(a)}^{-1}$
\end{enumerate}
\begin{proof}
$h(e)=h(e*e)=h(e)*h(e)$

$h(e)$ - идемпотент в $\mathcal{H}$, следовательно $h(e)=e$
\begin{multline*}
h(a^{-1})=h(a^{-1})*e=h(a^{-1})*h(a)*(h(a))^{-1}=\\
h(a^{-1}*a)*(h(a))^{-1}=h(e)*(h(a))^{-1}=e*(h(a))^{-1}=(h(a))^{-1}
\end{multline*}
\end{proof}
\end{cnsq}

\begin{dfn}[Порождённая конгруэнтность]
Конгруэнтность порождённая $h$ - если $a\equiv b \Leftrightarrow h(a)=h(b)$ - конгруэнтность, то $h[A]= \bigslant{A}{\equiv}$ 
\end{dfn}

\begin{thm}
Если $h:G\rightarrow H$ - гомоморфизм, $\equiv$ - конгруэнтность порождённая $h$, то классы эквивалентные $e$ в $G$ являются нормальными подгруппами
\begin{proof}
Пусть $a,b\in f\Rightarrow ab^{-1}\in f$, $a\equiv e$, $b\equiv e$, $b^{-1}\equiv e^{-1}\equiv e$, $ab^{-1}\equiv ee\equiv e$

$$a\{b\in \mathcal{G}:b\equiv e\}\ni c$$
$$aba^{-1}\in\{b\in \mathcal{G}:b\equiv e\}a\ni c$$

$$c=ab=abe=aba^{-1}a$$
$$b\equiv e \quad a\equiv a\quad a^{-1}\equiv a^{-1}$$
$$aba^{-1}\equiv aea^{-1}=e$$
$$aba^{-1}\equiv e$$
$$aba^{-1}a=abe=ab=c$$
\end{proof}
"И в обратную сторону". Хотя я в душе не знаю как в эту получилось.
\end{thm}

\begin{dfn}[Ядро подгруппы]
Ядро подгруппы - множество элементов эквивалентных $e$. $\operatorname{Ker}h$
\end{dfn}

\begin{thm}
$G$ - группа, $H$ - нормальная подгруппа, $a\equiv b \Leftrightarrow$ $a$ и $b$ принадлежат одному левому классу, то $\equiv$ - конгруэнтность
\begin{proof}
Пусть $a\equiv b$, $c\equiv d$, надо доказать
\begin{enumerate}
\item $ac\equiv bd$
\item $a^{-1}\equiv b^{-1}$ (зачем)
\end{enumerate}
\begin{enumerate}
\item 
\begin{align*}
a,b&\in x\mathcal{H} & a&=xh_a,  b=xh_b\\
c,d&\in y\mathcal{H} & c&=yh_c,  d=yh_d
\end{align*}
$ac=xh_a\cdot yh_c$, $h_ay=yh'$, $h_ay\in \mathcal{H}y=y\mathcal{H}$
$$
\begin{rcases}
ac=xh_ayh_c=xy\underbrace{h'h_c}_{\in\mathcal{H}}\in xy\mathcal{H}\\
bd=xh_byh_d=xy\underbrace{h''h_d}_{\in\mathcal{H}}\in xy\mathcal{H}
\end{rcases}
\text{эквивалентные}
$$
$h_by=yh''$, $h_by\in \mathcal{H}y=y\mathcal{H}$
\item
\begin{align*}
&h_a & &h_b\\
&h^{-1}_a & &h^{-1}_b\\
&\mathcal{H}x^{-1} & &\mathcal{H}x^{-1}
\end{align*}
$a^{-1},b^{-1}\in x^{-1}\mathcal{H}$
\end{enumerate}
\end{proof}
\end{thm}

\begin{dfn}[щито]
$\mathcal{G}$ - группа, $\mathcal{H}$ - нормальная подгруппа, $\equiv$ - отношение конгруэнтности. Тогда $\bigslant{\mathcal{G}}{\equiv}=\bigslant{\mathcal{G}}{\mathcal{H}}$
\end{dfn}

\begin{cnsq}
Если $h:\mathcal{G}\rightarrow \mathcal{H}$ - гомоморфизм, тогда $h[\mathcal{G}]=\bigslant{\mathcal{G}}{\Ker h}$
\begin{proof}
$h[\mathcal{G}]=\bigslant{\mathcal{G}}{\equiv}=\bigslant{\mathcal{G}}{\Ker h}$
\end{proof}
\end{cnsq}

\begin{exm}
$$\mathcal{D}_3=\{e,r_1,r_2,s_1,s_2,s_3\}$$
$\langle r_1\rangle$ - подгруппа вращений

$\langle r_1\rangle$

$S_1\langle r_1\rangle$

\begin{table}[h]
\centering
\caption*{Таблица умножения (ЧЕГО???)}
\renewcommand*{\arraystretch}{1.4}
\begin{tabular}{c|c|c}
  & $\langle r_1\rangle$ & $S_1\langle r_1\rangle$ \\ \hline
$\langle r_1\rangle$ & $\langle r_1\rangle$ & $S_1\langle r_1\rangle$ \\ \hline
$S_1\langle r_1\rangle$ & $S_1\langle r_1\rangle$ & $\langle r_1\rangle$  \\ 
\end{tabular}
\end{table}
\end{exm}

\begin{exm}
$(\mathbb{R},+)\supseteq (\mathbb{Z},+)$

$a+\mathbb{Z}$

$ba\in \mathbb{Z}$

$a+\mathbb{Z}=b+\mathbb{Z}$

$a\in [0,1)$

$(a+\mathbb{Z})+(b+\mathbb{Z})=(a+b)=(a+b)\mod 1$

$\mathbb{C}_1=\{z\in\mathbb{C},\vert z\vert=1\}$, $(\mathbb{C}_1,\cdot)$

$h(x)=e^{2nix}$

$x\in \mathbb{R}=e^{2nix}\in \mathbb{C}_1$

$h(x+y)=e^{2ni(x+y)}=e^{2nix}e^{2niy}=h(x)h(y)$

$h:(\mathbb{R},+)\rightarrow (\mathbb{C},\cdot)$

$r\in \Ker h \Leftrightarrow r\equiv e$

$h(r)=h(e)$

$h(r)=h(0)$

$e^{2nix}=e^{2nix}=1$

$e^{2nix}=2n\cdot k, k\in \mathbb{Z}$

$r\in \mathbb{Z}$

$\Ker h\in \mathbb{Z}$
\end{exm}
\end{document}