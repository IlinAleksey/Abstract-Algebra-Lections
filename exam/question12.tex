\documentclass[../main/document.tex]{subfiles}

\begin{document}
\section{Кольца, тела, поля. Делители нуля. Тело кватернионов}
\begin{dfn}[Кольцо]
Кольцо - алгебра сигнатуры
$$(+^{(2)},0^{(0)},{{}^{-}}^{(1)},\cdot^{(2)})$$
обладающее свойствами:
\begin{enumerate}
\item $(a+b)+c=a+(b+c)$
\item $a+0=a$
\item $a+(-a)=0$
\item $a+b=b+a$
\item $a(b+c)=ab+ac$
\end{enumerate}
\end{dfn}
\begin{dfn}[Ассоциативное кольцо]
Кольцо с ассоциативностью умножения $(ab)c=a(bc)$
\end{dfn}
\begin{dfn}[Кольцо с единицей]
Кольцо, в котором существует элемент $1$, такой что $a\cdot 1=1\cdot a=a$
\end{dfn}
\begin{dfn}[Коммутативное кольцо]
Кольцо с коммутативностью умножения $ab=ba$
\end{dfn}
\begin{dfn}[Кольцо с делением]
Если для любого элемента кольца $a\,(a\neq 0))$ существует $b:ab=1$, то такое кольцо называется кольцом с делением
\end{dfn}
\begin{dfn}[Тело]
Тело - ассоциативное, коммутативное кольцо с делением
\end{dfn}
\begin{dfn}[Поле]
Поле - ассоциативное, коммутативное кольцо с делением и единицей
\end{dfn}
\begin{exm}[Примеры колец]

\end{exm}
\begin{thm}
Для любых элементов кольца $a,b$ справедливы следующие утверждения:
\begin{enumerate}
\item $a0=0a=0$
\item $(-a)b=a(-b)=-(ab)$
\end{enumerate}
\begin{proof}

\end{proof}
\end{thm}

\begin{cnsq}
В кольце с $1$ ноль необратим.
\end{cnsq}
\begin{dfn}[Делитель нуля]
Пусть $a\cdot b=0\, a,b\neq 0$, тогда $a$ - левый делитель нуля, $b$ - правый делитель нуля.
\end{dfn}
\begin{exm}[Пример делителей нуля]

\end{exm}
\begin{thm}
Делители нуля необратимы
\begin{proof}

\end{proof}
\end{thm}
\begin{dfn}[Идемпотент кольца]
Такие элементы кольца, для которых выполняется $a=a^2$
\end{dfn}
\begin{thm}
Идемпотенты - делители нуля
\begin{proof}

\end{proof}
\end{thm}
\begin{dfn}[Тело кватернионов]

\end{dfn}
\end{document}