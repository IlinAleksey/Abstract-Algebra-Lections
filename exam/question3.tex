\documentclass[../main/document.tex]{subfiles}

\begin{document}
\section{Гомоморфизмы, гомоморфные образы,\\ конгруэнтности, фактор-алгебры}
\begin{dfn}[Гомоморфизм]\label{homomorphism}
Отображение $f \colon G_1 \to G_2$ называется гомоморфизмом групп $(G_1,*), ~(G_2,\times)$, если оно одну групповую операцию переводит в другую: $f(a*b)=f(a)\times f(b),\, a,b\in G_1$.
\end{dfn}
\begin{dfn}[Мономорфизм]
Инъективный (разнозначный) гомоморфизм
\end{dfn}
\begin{exm}[Пример на мономорфизм]

\end{exm}
\begin{dfn}[Эпиморфизм]
сюръективный гомоморфизм
\end{dfn}
\begin{exm}[Пример на Эпиморфизм]

\end{exm}
\begin{dfn}[Изоморфизм]
взаимно однозначный (биективный) гомоморфизм
\end{dfn}
\begin{exm}[Пример на Изоморфизм]

\end{exm}
\begin{dfn}[Эндоморфизм]
гомоморфизм в само множество
\end{dfn}
\begin{exm}[Пример на Эндоморфизм]

\end{exm}
\begin{dfn}[Автоморфизм]
взаимно однозначный гомоморфизм в само множество
\end{dfn}
\begin{exm}[Пример на Автоморфизм]

\end{exm}
\begin{dfn}[Гомоморфный образ]
Образ гомоморфизма
\end{dfn}
\begin{exm}[Пример на гомоморфный образ]

\end{exm}
\begin{dfn}[Конгруэнтность]
Отношение эквивалентности (рефликсивность, симметричность, транзитивность), сохраняющееся при основных операциях, то есть 
$$a_1\equiv a_2,\, b_1\equiv b_2\Rightarrow a_1\cdot b_1\equiv a_2\cdot b_2$$
\end{dfn}
\begin{dfn}[Фактор-алгебра]
Множество классов эквивалентности по отношению к конгруэнтности
\end{dfn}


\end{document}