\documentclass[../main/document.tex]{subfiles}

\begin{document}
\section{Полугруппы и моноиды. Идемпотенты, сократимые и обратимые элементы.}
\begin{dfn}[Полугруппа]
Полугруппа - многообразие заданное множеством
$$(x*y)*z=x*(y*z)$$
\end{dfn}
\begin{exm}[Примеры полугрупп]
\end{exm}
\begin{thm}
Значение терма не зависит от расстановки скобок (Ассоциативный закон)
$$t=t_1*t_2=(a_{1}a_{2}...a_{m})(a_{m+1}...a_n)=a_{1}a_{2}...a_{n}$$
\begin{proof}
Индукция по длине $t$

Базис: $n=1$, нет скобок

Шаг: для $n-1$ верно, тогда
\begin{enumerate}
\item $m=n-1$
$$t=t_1*a_n=(a_{1}a_{2}...a_{m})*a_n=a_{1}a_{2}...a_{n}$$
\item $1\leq m\leq n-1$
\begin{multline*}
t=t_1*t_2=(a_{1}a_{2}...a_{m})(a_{m+1}...a_n)=(a_{1}a_{2}...a_{m})(a_{m+1}...a_{n-1})a_n
\end{multline*}
Так как длина $(a_{1}a_{2}...a_{m})(a_{m+1}...a_{n-1})$ равна $n-1$ то выполняется индукционное предположение и
$$(a_{1}a_{2}...a_{m})(a_{m+1}...a_{n-1})=(a_{1}a_{2}...a_{n-1})$$
соотвественно
$$(a_{1}a_{2}...a_{m})(a_{m+1}...a_{n-1})a_n=
(a_{1}a_{2}...a_{n-1})a_n=a_{1}a_{2}...a_{n}$$
\end{enumerate}
\end{proof}
\end{thm}
\begin{dfn}[Нейтральный элемент]
$e_l$ называется \textbf{нейтральным слева} в полугруппе, если $e_l*a=a$ для всех $a$,
$e_r$ называется \textbf{нейтральным справа} в полугруппе, если $a*e_r=a$ для всех $a$,
$e$ - нейтральный слева и справа
\end{dfn}
\begin{exm}[Примеры нейтрального элемента]

$(\omega,+)$ - $0$, $(\omega,\cdot)$ - $1$, $(\omega,max)$ - $0$, $(\omega,min)$ - нет нейтрального
\end{exm}
\begin{thm}
Если существуют нейтральный слева и нейтральный справа то они равны
\begin{proof}
$$e_l=e_l*e_r=e_r$$
\end{proof}
\end{thm}
\begin{cnsq}
Если нейтральный элемент существует, то он единственный.
\end{cnsq}
\begin{dfn}[Моноид]
Моноид - полугруппа с нейтральным элементом ИЛИ

Моноид - это элементы многообразия, которые определяются равенствами
$$
\begin{cases}
x*(y*z)=(x*y)*z\\
x*e=x\\
e*x=x
\end{cases}$$
\end{dfn}
\begin{exm}[Примеры моноидов]

$(\omega,+,0)$, $(\omega,\cdot,1)$, $(\omega,max,0)$

$A^A$ - множество одноместных функций из $A$ в $A$
$h=f\circ g$, если $h(a)=g(f(a))$ для любого $a\in A$

Доказать что $(A^A,\circ)$ - моноид
\begin{proof}
$e(a)=a$ для всех $a$, тогда
\begin{equation*}
\begin{rcases}
(e\circ f)(a) &= f(e(a))= f(a) \\
(f\circ e)(a) &= e(f(a))= f(a) 
\end{rcases}
e\circ f=f\circ e=f
\end{equation*}
$e$ - нейтральный элемент

$$((f\circ g)h)(a)=h(f\circ g)(a)=h(g(f(a)))$$
$$(f(g\circ h))(a)=(g\circ h)(f(a))=h(g(f(a)))$$
$$((f\circ g)h)(a)=(f(g\circ h))(a)$$
Выполняется ассоциативность, соответственно $(A^A,\circ,e)$ - моноид

\end{proof}
\end{exm}
\begin{dfn}[Идемпотент]
Идемпотент - элемент моноида $a$, такой что $a^2=a$
\end{dfn}

\begin{exm}[Примеры идемпотентов]
$(\omega;+)$ - $0$
\end{exm}

\begin{dfn}[Обратный элемент]$ $\\

$b_l$ - левый обратный для элемента $a$, если $b_l*a=e$,

$b_r$ - правый обратный для элемента $a$, если $a*b_l=e$,

$b$ - обратный для элемента $a$, если $b*a=a*b=e$
\end{dfn}

\begin{dfn}[Обратимый элемент]
Элемент, для которого существует обратный
\end{dfn}
\begin{exm}
Пример чего-то:
Доказать что множество функций этого вида замкнуты относительно композиции:
$$f(x)=
\begin{cases}
ax & \text{при } x<b\\
ab & \text{при } x\geq b
\end{cases}
$$
\begin{tikzpicture}
% horizontal axis
\draw[->] (0,0) -- (6,0) node[anchor=north] {$x$};
% vertical axis
\draw[->] (0,0) -- (0,5) node[anchor=east] {$y$};

% labels
\draw	(0,0) node[anchor=north] {0}
		(1.8,0) node[anchor=north] {b}
		(4,0) node[anchor=north] {1}
		(0,4) node[anchor=east] {1};
% graph
\draw[thick] (0,0) -- (1.8,4)  -- (6,4);

% dotted lines
\draw[dotted] (1.8,0) -- (1.8,5);
\draw[dotted] (4,0) -- (4,5);
\draw[dotted] (0,4) -- (6,4);
\end{tikzpicture}
\begin{proof}

\end{proof}

\end{exm}

\begin{exm}[Пример изоморфизма]
Доказать
$$(P(A\cup B);\cup,\cap)\cong (P(A);\cup,\cap)\times (P(B);\cup,\cap)$$
где $P(A)$ - множество всех подмножеств множества $A$
\begin{proof}
Надо доказать
$$h(x_1\cup x_2)=h(x_1)\cup h(x_2)$$
$$h(x_1\cap x_2)=h(x_1)\cap h(x_2)$$
и $h$ - биекция

По сути функция $h$ должна выдавать пару, первая часть которой состоит из элементов $A$, вторая из $B$
\end{proof}
\end{exm}
\begin{exm}[Пример полугруппы]
Является ли $(\omega,\Nod())$ полугруппой
\begin{proof}
Предположим что является, надо доказать
$$\Nod(\Nod(x,y),z)=\Nod(x,\Nod(y,z))$$
\begin{enumerate}
\item $\Rightarrow$
Пусть $d: d|\Nod(x,y), d|z$

Надо доказать $d|\Nod(y,z)$, $d|x$

$$d|\Nod(x,y)\Rightarrow d|x$$
$$d|\Nod(x,y)\Rightarrow d|y$$
$$d|x,d|y\Rightarrow d|\Nod(y,z)$$
\item $\Leftarrow$ также
\end{enumerate}
\end{proof}
\end{exm}
\begin{exm}[Построение моноидов]
Построить все моноиды из двух элементов $\{e,x\}$

$$A_1=(\{e,x\};*_1), A_2=(\{e,x\};*_2)$$

\begin{table}[h]
\centering
\caption*{Таблица умножения $(*_1)$}
\renewcommand*{\arraystretch}{1.4}
\begin{tabular}{|l|l|l|}
\hline
  & $e$ & $x$  \\ \hline
$e$ & $e$ & $x$  \\ \hline
$x$ & $x$ & $e$ \\ 
\hline
\end{tabular}
\end{table}

\begin{table}[h]
\centering
\caption*{Таблица умножения $(*_2)$}
\renewcommand*{\arraystretch}{1.4}
\begin{tabular}{|l|l|l|}
\hline
  & $e$ & $x$  \\ \hline
$e$ & $e$ & $x$  \\ \hline
$x$ & $x$ & $x$ \\ 
\hline
\end{tabular}
\end{table}

Доказать их ассоциативность: $a*(b*c)=(a*b)*c$
\begin{enumerate}
\item $a=e$
$$e*(b*c)=b*c=(e*b)*c$$
\item $b=e$ также
\item $c=e$ также
\item $a=b=c=x$
$$x*(x*x)=x*e=e*x=(x*x)*x$$
\end{enumerate}
Все остальные моноиды или изоморфны или тривиальны
\end{exm}

\begin{thm}
Если в конечном моноиде каждый элемент имеет левый обратный, то существует правый обратный
\begin{proof}
Предположим обратное: Если в конечном моноиде каждый элемент имеет левый обратный, то хотя бы для одного не существует правый обратный: $ab_r\neq e$ для всех $b_r$

НЕ ДОКАЗАНО
\end{proof}
\end{thm}

\begin{dfn}[Сократимый элемент]
Сократимый слева (справа) - такой элемент моноида, что из $ax=ay$ ($xa=ya$) следует $x=y$ 
\end{dfn}
\begin{exm}[Пример сократимого элемента]
$(\mathbb{Z},+,0)$, $x+a=y+a\Rightarrow x=y$
\end{exm}
\begin{thm}
Неединичные идемпотенты несократимы
\begin{proof}
$a\cdot a=a=e\cdot a$ но $a\neq e$, соответственно $a$ несократим справа,
$a\cdot a=a=a\cdot e$ но $a\neq e$, соответственно $a$ несократим слева

$a$ несократим
\end{proof}
\end{thm}
\begin{thm}
Все обратимые слева(справа) элементы сократимы слева(справа)
\begin{proof}
Пусть $a$ - обратимый слева, тогда
$ax=ay\Rightarrow b_lax=b_lay\Rightarrow ex=ey\Rightarrow x=y$, следовательно $a$ - сократимый слева
\end{proof}
\end{thm}
\begin{exm}[Пример обратимого элемента]
$(\mathbb{Z}^+,\cdot,1)$, обратимый только $1$, сократимы все. (Какой к половым органам это пример?)
\end{exm}
\end{document}