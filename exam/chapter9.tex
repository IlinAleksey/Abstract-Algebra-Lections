\documentclass[../main/document.tex]{subfiles}

\begin{document}
\section{Кольца и поля}
\begin{dfn}[Кольцо]
Кольцо - алгебра сигнатуры
$$(+^{(2)},0^{(0)},{{}^{-}}^{(1)},\cdot^{(2)})$$
обладающее свойствами:
\begin{enumerate}
\item $(a+b)+c=a+(b+c)$
\item $a+0=a$
\item $a+(-a)=0$
\item $a+b=b+a$
\item $a(b+c)=ab+ac$
\end{enumerate}
\end{dfn}
\begin{dfn}[Ассоциативное кольцо]
Кольцо с ассоциативностью умножения $(ab)c=a(bc)$
\end{dfn}
\begin{dfn}[Кольцо с единицей]
Кольцо, в котором существует элемент $1$, такой что $a\cdot 1=1\cdot a=a$
\end{dfn}
\begin{dfn}[Коммутативное кольцо]
Кольцо с коммутативностью умножения $ab=ba$
\end{dfn}
\begin{dfn}[Кольцо с делением]
Если для любого элемента кольца $a\,(a\neq 0))$ существует $b:ab=1$, то такое кольцо называется кольцом с делением
\end{dfn}
\begin{dfn}[Тело]
Тело - ассоциативное, коммутативное кольцо с делением
\end{dfn}
\begin{dfn}[Поле]
Поле - ассоциативное, коммутативное кольцо с делением и единицей
\end{dfn}
\begin{exm}[Примеры колец]

\end{exm}
\begin{thm}
Для любых элементов кольца $a,b$ справедливы следующие утверждения:
\begin{enumerate}
\item $a0=0a=0$
\item $(-a)b=a(-b)=-(ab)$
\end{enumerate}
\begin{proof}

\end{proof}
\end{thm}

\begin{cnsq}
В кольце с $1$ ноль необратим.
\end{cnsq}
\begin{dfn}[Делитель нуля]
Пусть $a\cdot b=0\, a,b\neq 0$, тогда $a$ - левый делитель нуля, $b$ - правый делитель нуля.
\end{dfn}
\begin{exm}[Пример делителей нуля]

\end{exm}
\begin{thm}
Делители нуля необратимы
\begin{proof}

\end{proof}
\end{thm}
\begin{dfn}[Идемпотент кольца]
Такие элементы кольца, для которых выполняется $a=a^2$
\end{dfn}
\begin{thm}
Идемпотенты - делители нуля
\begin{proof}

\end{proof}
\end{thm}
\begin{dfn}[Целостное кольцо]
Ассоциативное, коммутативное кольцо с единицей без делителей нуля
\end{dfn}
\begin{thm}
Конечное целое кольцо ?????
\begin{proof}

\end{proof}
\end{thm}
\begin{thm}
Каждое целостное кольцо может быть достроено до поля
\begin{proof}

\end{proof}
\end{thm}
\begin{dfn}[Гомоморфизм колец]
$h:R\to S$ - гомоморфизм, определённый так: $a\equiv b\Leftrightarrow h(a)=h(b)$
\end{dfn}
\begin{dfn}[Ядро кольца]
$h:R\to S$ - гомоморфизм, тогда ядро кольца $\Ker h=\{a\in R:h(a)=0\}$ 
\end{dfn}
\begin{thm}
Ядро кольца - подкольцо
\end{thm}
\begin{dfn}[Идеал]
$R$ - кольцо, $\mathcal{I}\subseteq R$ - идеал (левый, правый, двусторонний), если
\begin{enumerate}
\item $\mathcal{I}$ - подкольцо
\item для любого $x\in R$ $x\mathcal{I}\subseteq\mathcal{I}$ (левый идеал), $\mathcal{I} x\subseteq\mathcal{I}$ (правый идеал)
\end{enumerate}
\end{dfn}
\begin{exm}[Пример идеалов]

\end{exm}
\begin{thm}
$R$ - ассоциативное кольцо с единицей или $R$ - тело или $R$ тогда и только тогда когда в $R$ Нет других идеалов, кроме $\{0\}$ и $R$
\end{thm}
\begin{dfn}[Булевое кольцо]

\end{dfn}
\begin{thm}
Пусть $I$ - двухсторонний идеал в $R$, тогда отношение $\equiv: x\equiv y \Leftrightarrow x-y\in I$ является конгруэнтностью
\begin{proof}

\end{proof}
\end{thm}
\begin{cnsq}
Существует фактор-алгебра $\bigslant{R}{\equiv}$, такая что ??????????????????
\end{cnsq}

\begin{cnsq}
$I=\Ker h$, где $h:R\to\bigslant{R}{\equiv} $
\begin{proof}

\end{proof}
\end{cnsq}
\begin{dfn}[Простой идеал]
Пусть $R$ - ассоциативное, коммутативное кольцо с единицей, тогда $I$ - простой идеал, если $ab\in I\Leftrightarrow a\in I$ или $b\in I$
\end{dfn}
\begin{dfn}[Максимальный идеал]
Пусть $R$ - ассоциативное, коммутативное кольцо с единицей, тогда $I$ - максимальный идеал, если для любого идеала $J: I\subseteq J, I\neq J$ выполняется $J=R$
\end{dfn}
\begin{dfn}[Главный идеал]
Пусть $R$ - ассоциативное, коммутативное кольцо с единицей, тогда $I$ - главный идеал, если для некоторого $a\in R$ $I=aR$
\end{dfn}
\begin{exm}[??????]

\end{exm}
\begin{lemma}
Если $I$ и $J$ - идеалы, то $I+J$ тоже идеал
\begin{proof}

\end{proof}
\end{lemma}
\begin{thm}
Пусть $R$ - ассоциативное, коммутативное кольцо с единицей, $I$ - идеал, тогда
\begin{enumerate}
\item $I$ - простой идеал $\Leftrightarrow$ $\bigslant{R}{I}$ - целостное
\item $I$ - максимальный идеал $\Leftrightarrow$ $\bigslant{R}{I}$ - поле
\end{enumerate}
\begin{proof}

\end{proof}
\end{thm}
\begin{dfn}[Евклидово кольцо]
$R$ - ассоциативное, коммутативное кольцо с единицей, $R$ - евклидово, если для каждого элемента $a$ этого кольца существует его норма $\Vert a\Vert$.
\end{dfn}
\begin{dfn}[Евклидова норма]
Это некоторая функция элемента кольца, такая что
\begin{enumerate}
\item $\Vert a\Vert \in \omega$
\item если $a,b\neq 0$, то $\Vert ab\Vert \geq \max(\Vert a\Vert,\Vert b\Vert)$
\item если $$
\end{enumerate}
\end{dfn}
\end{document}