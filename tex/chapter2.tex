\documentclass[../main/document.tex]{subfiles}

\begin{document}
\section{Изоморфизм}
\begin{dfn}
\textbf{Изоморфизм} - Пусть $\Sigma$ - сигнатура, $\mathpzc{A}=(A,I)$, $\mathpzc{B}=(B,J)$ - \\
 универсальные алгебры сигнатуры $\Sigma$, тогда изоморфизм между $\mathpzc{A}$ и $\mathpzc{B}$ - это $h:\mathpzc{A}\rightarrow \mathpzc{B}$ - биективная функция, которая удовлетворяет следующему условию:
$$h(I(f_i)(a_1,...,a_n))=J(f_i)(h(a_1),...,h(a_n))$$
для любых $a_1,...,a_n$ и $f_i\in \Sigma$
\end{dfn}

\begin{exm}
Пример изоморфизма: пусть $\Sigma=(f^{(2)})$, $\mathpzc{A}=(\mathbb{R},+)$, $\mathpzc{B}=(\mathbb{R},\cdot)$

Надо доказать: 
$$h(a_1+a_2)=h(a_1)\cdot h(a_2)$$
$a_1,a_2\in \mathbb{R}$

Пусть $h(x)=e^x$, тогда
$$h(a_1+a_2)=e^{a_1+a_2}=e^{a_1}\cdot e^{a_2}=h(a_1)\cdot h(a_2)\blacksquare$$
\end{exm}

\begin{thm}\label{isomorphism-inverse}
$h$ - изоморфизм между $A$ и $B$, то $h^{-1}$ - изоморфизм между $B$ и $A$
\begin{proof} пусть $b_1,...,b_{n_i}\in B$, тогда надо доказать
$$h^{-1}(J(f_i)(b_1,...,b_{n_i}))=I(f_i)(h^{-1}(b_1),...,h^{-1}(b_{n_i}))$$
Так как $b_1=h(a_1),...,b_{n_i}=h(a_{n_i})$,
$$I(f_i)(h^{-1}(b_1),...,h^{-1}(b_{n_i}))=I(f_i)(h^{-1}(h(a_1)),...,h^{-1}(h(a_{n_i})))=I(f_i)(a_1,...,a_{n_i})$$
По определению изоморфизма
$$h^{-1}(J(f_i)(b_1,...,b_{n_i}))=h^{-1}(h(I(f_i)(a_1,...,a_{n_1})))=I(f_i)(a_1,...,a_{n_1})$$
Из этих двух равенств следует то, что надо доказать
\end{proof}
\end{thm}

\begin{dfn}
Системы, между которыми существует изоморфизм называют \textbf{изоморфными}
$$\mathpzc{A}\simeq\mathpzc{B}$$
операции в изоморфных системах обладают одними и теми же свойствами
\end{dfn}

\begin{dfn}
$t(x_1,...,x_n)$ - терм $t$ не содержит других переменных кроме $x_1,...,x_n$
\end{dfn}
\begin{dfn}
Пусть $\mathpzc{A}$ - алгебра, $a_1,...,a_n$ - элементы алгебры $\mathpzc{A}$, тогда
$$t(a_1,...,a_n)=\sigma(t), \sigma(x_1)=a_1,...,\sigma(x_n)=a_n$$
\end{dfn}
\begin{thm}
$h$ - изоморфизм между $\mathpzc{A}=(A,I)$ и $\mathpzc{B}=(B,J)$, то для любого терма $t(x_1,...,x_n)$ и любых $a_1,...,a_n$ выполняется
$$h(t^{\mathpzc{A}}(a_1,...,a_n))=t^{\mathpzc{B}}(h(a_1),...,h(a_n))$$
\end{thm}
\begin{proof}
Индукция по построению терма $t$
\begin{enumerate}
\item $t=x$
$$t^{\mathpzc{A}}(a)=a\Leftrightarrow h(t^{\mathpzc{A}}(a))=h(a)\Leftrightarrow t^{\mathpzc{B}}(h(a))=h(a)$$
\item $t=c$
$$\sigma(c)=I(c)=J(c)\Rightarrow t^{\mathpzc{A}}=I(c), t^{\mathpzc{B}}=J(c)\Rightarrow h(I(c))=J(c)$$
по определению гомоморфизма
\item $t=f(t_1,...,t_k)$
\begin{multline*}
h(t^{\mathpzc{A}}(a_1,...,a_n))=\\
h(I(f)(t^{\mathpzc{A}}_{1}(a_1,...,a_n),...,t^{\mathpzc{A}}_{k}(a_1,...,a_n)))=\\
J(f)(h(t^{\mathpzc{A}}_{1}(a_1,...,a_n)),...,h(t^{\mathpzc{A}}_{k}(a_1,...,a_n)))=\\
J(f)(t^{\mathpzc{B}}_{1}(h(a_1),...,h(a_n)),...,t^{\mathpzc{B}}_{k}(h(a_1),...,h(a_n))=\\
t^{\mathpzc{B}}(h(a_1),...,h(a_n))
\end{multline*}
\end{enumerate}
\end{proof}
\begin{exm}
Доказать что $\mathpzc{A}=(\mathbb{R};\cdot)\not\cong\mathpzc{B}=(\mathbb{R}^{+};\cdot)$
\begin{proof}
Предположим что существует изоморфизм $h:\mathpzc{A}\rightarrow \mathpzc{B}$, тогда

$h(0)=x, x\in \mathbb{R}^{+}$
$$x=h(0)=h(0\cdot 0)=h(0)\cdot h(0)=x^2$$
$$x=x^2\Rightarrow x=1$$

$h(1)=y, y\in \mathbb{R}^{+} $
$$y=h(1)=h(1\cdot 1)=h(1)\cdot h(1)=y^2$$
$$y=y^2\Rightarrow y=1$$

$h(0)=1=h(1)$ - противоречие ($h$ не биективна). Утверждение не верно.
\end{proof}
\end{exm}
\begin{exm}
Доказать что $\mathpzc{A}=(\mathbb{R};+)\not\cong\mathpzc{B}=(\mathbb{R};\cdot)$
\begin{proof}
Предположим что существует изоморфизм $h:\mathpzc{B}\rightarrow \mathpzc{A}$, тогда

$h(0)=x, h(1)=y; x,y\in \mathbb{R}$
$$x=h(0)=h(0\cdot 0)=h(0)+ h(0)=2x\Rightarrow x=2x=0$$
$$y=h(1)=h(1\cdot 1)=h(1)+ h(1)=2y\Rightarrow y=2y=0$$
Противоречие ($h$ должно быть биекцией)
\end{proof}
\end{exm}

\begin{exm}
Доказать что $\mathpzc{A}=(\mathbb{R};\cdot)\cong\mathpzc{B}=(\mathbb{C};\cdot)$
\begin{proof}
Предположим что существует изоморфизм $h:\mathpzc{B}\rightarrow \mathpzc{A}$, тогда

$h(x)=-1; x\in \mathbb{C},-1\in \mathbb{R}$

\end{proof}
\end{exm}

\begin{exm}
Доказать что $\mathpzc{A}=(\mathbb{Z};{\operatorname{min}}^{(2)})\not\cong\mathpzc{B}=(\mathbb{Z};{\operatorname{max}}^{(2)})$
\begin{proof}

\end{proof}
\end{exm}

\begin{exm}
Доказать что $\mathpzc{A}=(\omega;+)\not\cong\mathpzc{B}=(\omega^+;\cdot)$
\begin{proof}

\end{proof}
\end{exm}

\begin{exm}
Доказать что $\mathpzc{A}=(\mathbb{Q};+)\not\cong\mathpzc{B}=(\mathbb{Q}^+;\cdot)$
\begin{proof}

\end{proof}
\end{exm}

\begin{exm}
Доказать что $\mathpzc{A}=(\mathbb{Z};\cdot)\not\cong\mathpzc{B}=(\mathbb{G};\cdot)$
\begin{proof}

\end{proof}
\end{exm}
\end{document}
