\documentclass[../main/document.tex]{subfiles}

\begin{document}
\section{Подгруппы и моноиды}
\begin{dfn}
Подгруппа - многообразие заданное множеством
$$(x*y)*z=x*(y*z)$$
\end{dfn}
\begin{thm}
Значение терма не зависит от расстановки скобок (Ассоциативный закон)
$$t=t_1*t_2=(a_{1}a_{2}...a_{m})(a_{m+1}...a_n)=a_{1}a_{2}...a_{n}$$
\begin{proof}
Индукция по длине $t$

Базис: $n=1$, нет скобок

Шаг: для $n-1$ верно, тогда
\begin{enumerate}
\item $m=n-1$
$$t=t_1*a_n=(a_{1}a_{2}...a_{m})*a_n=a_{1}a_{2}...a_{n}$$
\item $1\leq m\leq n-1$
\begin{multline*}
t=t_1*t_2=(a_{1}a_{2}...a_{m})(a_{m+1}...a_n)=(a_{1}a_{2}...a_{m})(a_{m+1}...a_{n-1})a_n=\\
(a_{1}a_{2}...a_{n-1})a_n=a_{1}a_{2}...a_{n}
\end{multline*}
\end{enumerate}
\end{proof}
\end{thm}
\begin{dfn}
$e_l$ называется \textbf{нейтральным слева} в подгруппе, если $e_l*a=a$ для всех $a$,
$e_r$ называется \textbf{нейтральным справа} в подгруппе, если $a*e_r=a$ для всех $a$,
$e$ - нейтральный слева и справа
\end{dfn}
\begin{exm}
Примеры нейтрального элемента:
\end{exm}
\begin{thm}
Если существуют нейтральный слева и нейтральный справа то они равны
\begin{proof}
$$e_l=e_l*e_r=e_r$$
\end{proof}
\end{thm}
\begin{cnsq}
Если нейтральный элемент существует, то он единственный.
\end{cnsq}
\begin{dfn}
Моноид - подгруппа с нейтральным элементом
\end{dfn}
\begin{exm}
Примеры моноидов:
\end{exm}
\begin{dfn}
Свободный моноид - моноид, элементами которого являются конечные последовательности (строки) элементов носителя моноида. Свободный моноид на множестве $A\neq \emptyset$ это $\mathcal{A}=(A^*;\&)$
\end{dfn}
\begin{thm}
Любой моноид, порождённый элементами множества, на котором есть свободный моноид, является гомоморфным образом этого моноида
\begin{proof}
Пусть $A\neq \emptyset$, $\mathcal{A}=(A^*;\&)$,\\ $\mathcal{B}=(\left\{t^{\mathcal{B}}(a_1,...,a_n): a_1,...,a_n\in A\right\};*)$ и $h:\mathcal{A}\rightarrow \mathcal{B}$ - Гомоморфизм
$$h(a_1...a_n)=(a_1,...,a_n)^{\mathcal{B}}$$
$$h(\varepsilon)=e^{\mathcal{B}}$$
Надо доказать свойство гомоморфизма:
$$h(u\&v)=h(u)*h(v)$$
Пусть $u=a_1...a_n$, $v={a'}_1...{a'}_n$, тогда
$$h(u\&v)=h(uv)=h(a_1...a_n{a'}_1...{a'}_n)=(a_1...a_n{a'}_1...{a'}_n)^{\mathcal{B}}$$
\begin{multline*}
h(u)*h(v)=h(a_1...a_n)*h({a'}_1...{a'}_n)=\\
(a_1...a_n)^{\mathcal{B}}*({a'}_1...{a'}_n)^{\mathcal{B}}=(a_1...a_n{a'}_1...{a'}_n)^{\mathcal{B}}
\end{multline*}
Из этого следует что $h(u\&v)=h(u)*h(v)$
\end{proof}
\end{thm}
\begin{exm}
Примеры свободных моноидов и их гомоморфных образов:
\end{exm}

\begin{dfn}

Циклический моноид - моноид порождённый одним элементом. $<a>$ - циклический моноид, порождённый элементом $a$.

$e,a,a^1,a^2,a^3,...$ - элементы моноида $<a>$

\begin{enumerate}
\item $a^i\neq a^j$ при $i\neq j$

$h:<a>\rightarrow ({\left\{a\right\}}^*;\&)$, $h(a^i)=i$ - изоморфизм.
\item $a^i=a^j$ при $i\neq j$
$$k=i+(k-i)=i+y(j-i)+r$$
$$r=(k-i)mod(j-i)$$
$$r<j-i$$
тогда
\begin{multline*}
a^k=a^i\underbrace{a^{j-i}...a^{j-i}}_{y}a^r=\\
(a^ia^{j-i})\underbrace{a^{j-i}...a^{j-i}}_{y-1}a^r\stackrel{(a^ia^{j-i}=a^{i+j-i}=a^j=a^i)}{=}a^i\underbrace{a^{j-i}...a^{j-i}}_{y-1}a^r=\\
a^ia^r=a^{i+r} (r<j-i; i+r<j)
\end{multline*}
\end{enumerate}
\end{dfn} %% к чему весь этот список?

\begin{exm}\label{cyclical-exm}
Пример циклическокококого моноида:
$<a>=(\{e,a,...\};*)$

Таблица умножения $(*)$ -
\begin{table}[h]
\centering
\renewcommand*{\arraystretch}{1.4}
\begin{tabular}{|l|l|l|l|}
\hline
  & $e$ & $a$ & $a^2$ \\ \hline
$e$ & $a$ & $a$ & $a^2$ \\ \hline
$a$ & $a$ & $a^2$ & $a$ \\ \hline
$a^2$ & $a^2$ & $a$ & $a^2$ \\
\hline
\end{tabular}
\end{table}
\end{exm}

\end{document}
