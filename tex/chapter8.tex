\documentclass[../main/document.tex]{subfiles}

\begin{document}
\section{Гомоморфизмы группы}
\begin{thm}
Если $G$ и $H$ - группа, $h:G\rightarrow H$ и $h(a*b)=h(a)*h(b)$, то $h$ - гомоморфизм
\begin{proof}
$h(e)=h(e*e)=h(e)*h(e)$

$h(e)$ - идемпотент в $\mathcal{H}$, следовательно $h(e)=e$
\begin{multline*}
h(a^{-1})=h(a^{-1})*e=h(a^{-1})*h(a)*(h(a))^{-1}=\\
h(a^{-1}*a)*(h(a))^{-1}=h(e)*(h(a))^{-1}=e*(h(a))^{-1}=(h(a))^{-1}
\end{multline*}
\end{proof}
\end{thm}

\begin{dfn}
Конгруэнтность порождённая $h$ - если $a\equiv b \Leftrightarrow h(a)=h(b)$ - конгруэнтность, то $h[A]= \bigslant{A}{\equiv}$ 
\end{dfn}

\begin{thm}
Если $h:G\rightarrow H$ - гомоморфизм, $\equiv$ - конгруэнтность порождённая $h$, то классы эквивалентные $e$ в $G$ являются нормальными подгруппами
\begin{proof}
Пусть $a,b\in f\Rightarrow ab^{-1}\in f$, $a\equiv e$, $b\equiv e$, $b^{-1}\equiv e^{-1}\equiv e$, $ab^{-1}\equiv ee\equiv e$

$$a\{b\in \mathcal{G}:b\equiv e\}\ni c$$
$$aba^{-1}\in\{b\in \mathcal{G}:b\equiv e\}a\ni c$$

$$c=ab=abe=aba^{-1}a$$
$$b\equiv e \quad a\equiv a\quad a^{-1}\equiv a^{-1}$$
$$aba^{-1}\equiv aea^{-1}=e$$
$$aba^{-1}\equiv e$$
$$aba^{-1}a=abe=ab=c$$
\end{proof}
"И в обратную сторону". Хотя я в душе не знаю как в эту получилось.
\end{thm}

\begin{dfn}
Ядро подгруппы - множество элементов эквивалентных $e$. $\operatorname{Ker}h$
\end{dfn}

\begin{thm}
$G$ - группа, $H$ - нормальная подгруппа, $a\equiv b \Leftrightarrow$ $a$ и $b$ принадлежат одному левому классу, то $\equiv$ - конгруэнтность
\begin{proof}
Пусть $a\equiv b$, $c\equiv d$, надо доказать
\begin{enumerate}
\item $ac\equiv bd$
\item $a^{-1}\equiv b^{-1}$ (зачем)
\end{enumerate}
\begin{enumerate}
\item 
\begin{align*}
a,b&\in x\mathcal{H} & a&=xh_a,  b=xh_b\\
c,d&\in y\mathcal{H} & c&=yh_c,  d=yh_d
\end{align*}
$ac=xh_a\cdot yh_c$, $h_ay=yh'$, $h_ay\in \mathcal{H}y=y\mathcal{H}$
$$
\begin{rcases}
ac=xh_ayh_c=xy\underbrace{h'h_c}_{\in\mathcal{H}}\in xy\mathcal{H}\\
bd=xh_byh_d=xy\underbrace{h''h_d}_{\in\mathcal{H}}\in xy\mathcal{H}
\end{rcases}
\text{эквивалентные}
$$
$h_by=yh''$, $h_by\in \mathcal{H}y=y\mathcal{H}$
\item
\begin{align*}
&h_a & &h_b\\
&h^{-1}_a & &h^{-1}_b\\
&\mathcal{H}x^{-1} & &\mathcal{H}x^{-1}
\end{align*}
$a^{-1},b^{-1}\in x^{-1}\mathcal{H}$
\end{enumerate}
\end{proof}
\end{thm}

\begin{dfn}
$\mathcal{G}$ - группа, $\mathcal{H}$ - нормальная подгруппа, $\equiv$ - отношение конгруэнтности. Тогда $\bigslant{\mathcal{G}}{\equiv}=\bigslant{\mathcal{G}}{\mathcal{H}}$
\end{dfn}

\begin{cnsq}
Если $h:\mathcal{G}\rightarrow \mathcal{H}$ - гомоморфизм, тогда $h[\mathcal{G}]=\bigslant{\mathcal{G}}{\Ker h}$
\begin{proof}
$h[\mathcal{G}]=\bigslant{\mathcal{G}}{\equiv}=\bigslant{\mathcal{G}}{\Ker h}$
\end{proof}
\end{cnsq}

\begin{cnsq}
Для всякой нормальной группы верно $aH\cdot bH=abH$
\begin{proof}
$aH\cdot bH=aH\cdot Hb=aHb=abH$
\end{proof}
\end{cnsq}

\begin{exm}
$$\mathcal{D}_3=\{e,r_1,r_2,s_1,s_2,s_3\}$$
$\langle r_1\rangle$ - подгруппа вращений

$\langle r_1\rangle$

$S_1\langle r_1\rangle$

\begin{table}[h]
\centering
\caption*{Таблица умножения (ЧЕГО???)}
\renewcommand*{\arraystretch}{1.4}
\begin{tabular}{c|c|c}
  & $\langle r_1\rangle$ & $S_1\langle r_1\rangle$ \\ \hline
$\langle r_1\rangle$ & $\langle r_1\rangle$ & $S_1\langle r_1\rangle$ \\ \hline
$S_1\langle r_1\rangle$ & $S_1\langle r_1\rangle$ & $\langle r_1\rangle$  \\ 
\end{tabular}
\end{table}
\end{exm}

\begin{exm}
$(\mathbb{R},+)\supseteq (\mathbb{Z},+)$

$a+\mathbb{Z}$

$ba\in \mathbb{Z}$

$a+\mathbb{Z}=b+\mathbb{Z}$

$a\in [0,1)$

$(a+\mathbb{Z})+(b+\mathbb{Z})=(a+b)=(a+b)\mod 1$

$\mathbb{C}_1=\{z\in\mathbb{C},\vert z\vert=1\}$, $(\mathbb{C}_1,\cdot)$

$h(x)=e^{2nix}$

$x\in \mathbb{R}=e^{2nix}\in \mathbb{C}_1$

$h(x+y)=e^{2ni(x+y)}=e^{2nix}e^{2niy}=h(x)h(y)$

$h:(\mathbb{R},+)\rightarrow (\mathbb{C},\cdot)$

$r\in \Ker h \Leftrightarrow r\equiv e$

$h(r)=h(e)$

$h(r)=h(0)$

$e^{2nix}=e^{2nix}=1$

$e^{2nix}=2n\cdot k, k\in \mathbb{Z}$

$r\in \mathbb{Z}$

$\Ker h\in \mathbb{Z}$
\end{exm}

\begin{dfn}
$\mathcal{G}$ - группа, $A$ - множество, образующее группу, тогда определяющим соотношением называют равенство вида $t(a)=s(a)$, где $t,s$ - термы, $a\in A$
\end{dfn}

\begin{exm}
$A=\{a,b\}$, $a^2=b^2$, $a^3b=ba$
\end{exm}

\begin{dfn}
$A$ - множество элементов, $X$ - множество определяющих соотношений. Группа, порождённая $A$ и $X$ - $\mathcal{G}$ такач, что
\begin{enumerate}
\item образована при помощи $A$
\item в $\mathcal{G}$ выполняются все определяющие соотношения из $X$
\item любая группа $\mathcal{H}$, удовлетворяющая условиям 1 и 2 является гомоморфным множеством $\mathcal{G}$
\end{enumerate}
\end{dfn}

\begin{exm}
$$\mathcal{D}_3=\{e,r_1,r_2,s_1,s_2,s_3\}$$
$A=\{r_1,s_1\}$, $\langle A\rangle=\mathcal{D}_3$

$
\begin{sqcases}
r_1^3=e\\
r_1s_1=s_1r_1^2\\
s_1^2=e
\end{sqcases}$

$\mathcal{H}$ порождена $A$

$*$ - одноместная операция

$\mathcal{H}$ ?????? ??? слова, состоящие из $r_1,s_1,r_1^{-1},s_1^{-1}$, пусть в $\mathcal{H}$ выполнены определяющие соотношения $X$
\begin{align*}
r_1^3&=e & r_1^{-1}&=r_1^2 & r_1^{-1}&=r_1r_1\\
s_1^2&=e & s_1^{-1}&=s_1 & s_1^{-1}&=s_1
\end{align*}
$s_1...s_1r_1...r_1$

$s_1^nr_1^m$

$s_1^n=s_1^{n\mod 2}$

$r_1^m=r^{m\mod 3}$
\begin{table}[h]
\renewcommand*{\arraystretch}{1.4}
\begin{tabular}{|c c|}

\hline
 $r_1^0$ & $s_1r_1^0$  \\ 
$r_1^0$ & $s_1r_1^0$ \\ 
$r_1^0$ & $s_1r_1^0$  \\ 
\hline
\end{tabular}
\end{table}
\end{exm}

\begin{thm}
Для любого множества $A$ и множества определяющих соотношений $X$ существует группа, образованная $A$ и $X$
\begin{proof}
Пусть $A'=A\cup\{a{-1}:a\in A^\}$. Нужно проверить три свойства
\begin{enumerate}
\item Если $M$ - свободный моноид образованный $A'$($M$ - множество слов алфавита $A'$ с конкатенацией), $M'$ - моноид, порождённый $A'$, то $M'$ - гомоморфный образ $M$. $u,v\in M$, $u\equiv v\Leftrightarrow h(u)=h(v)$ для любого гомоморфизма $h:M\rightarrow \mathcal{G}$. $\mathcal{G}$ - группа, порождённая $A$ в которой ??? $X$.

Надо доказать что $\equiv$ является конгруэнтностью
\begin{enumerate}
\item $a\equiv a$
\item $a\equiv b\Rightarrow b\equiv a$
\item $a\equiv b, b\equiv c\Rightarrow a\equiv c$
\end{enumerate}
Пусть $a\equiv b$, $c\equiv d$, то есть $h(a)=h(b)$, $h(c)=h(d)$, тогда, так как $h$  является гомоморфизмом
$$h(ac)=h(a)h(c)=h(b)h(d)=h(bd)$$
следовательно $ac\equiv bd$ и $\equiv$ - конгруэнтность

Пусть группа $F=\bigslant{M}{\equiv}$, $\widehat{a}\in F$, $a=u_1...u_n$, $b=u_n^{-1}...u_1^{-1}$, $a,b\in M$

$h(a)=h(u_1)...h(u_n)$

$h(b)=h(u_n^{-1})...h(u_1^{-1})$

$h(ab)=h(u_1)...h(u_n)h(u_n^{-1})...h(u_1^{-1})=e$

$\widehat{a}\widehat{b}=\widehat{e}$

$F$ порождается $A$
\item Доказать $t(\overline{a})=s(\overline{a})\in X$

$h(t(a_1,...,a_n))=t(h(a_1),...,h(a_n))=s(h(a_1),...,h(a_n))=h(s(a_1,...,a_n))$
$t(\overline{a})\equiv s(\overline{a})\Rightarrow \widehat{t(\overline{a})}=widehat{s(\overline{a})}\Rightarrow t(\widehat{a_1},...,\widehat{a_n})=s(\widehat{a_1},...,\widehat{a_n}$
\item Из чего следует?
\end{enumerate}
и WTF в общем
\end{proof}
\end{thm}

\begin{exm}
Про пирамиду рубика. Конём.
\end{exm}

\begin{exm}
Дана "головоломка"

\begin{table}[H]
\renewcommand*{\arraystretch}{1.4}
\begin{tabular}{|c|c|}
\hline
1 & 2\\\hline
3 & 4\\
\hline
\end{tabular}
\end{table}
Построить группу $\mathcal{G}$


$a$ - перестановка двух столбцов

$b$ - перестановка строк

\begin{table}[H]
$e$:
\begin{tabular}{|c|c|}
\hline
1 & 2\\\hline
3 & 4\\
\hline
\end{tabular}
$a$:
\begin{tabular}{|c|c|}
\hline
2 & 3\\\hline
4 & 1\\
\hline
\end{tabular}
$b$:
\begin{tabular}{|c|c|}
\hline
3 & 4\\\hline
1 & 2\\
\hline
\end{tabular}
$ab$:
\begin{tabular}{|c|c|}
\hline
4 & 1\\\hline
2 & 3\\
\hline
\end{tabular}
\end{table}

$a^2=e$, $b^2=e$, $ab=ba$
\begin{table}[H]
\centering
\begin{tabular}{c|c|c|c|c}
 & $e$& $a$&$b$&$ab$\\\hline
$e$& $e$& $a$&$b$&$ab$\\\hline 
$a$& $a$& $e$&$ab$&$b$\\\hline 
$b$& $b$& $ba$&$e$&$a$\\\hline 
$ab$& $ab$& $b$&$a$&$e$\\ 
\end{tabular}
\end{table}

$\mathcal{G}=(\{e,a,b,ab\},\circ)$
\end{exm}

\begin{exm}
Таблица 8x8. Конём.
\end{exm}

\begin{exm}
$Z=1,-1$
\end{exm}

\begin{exm}
\end{exm}

\begin{exm}
\end{exm}

\begin{exm}
\end{exm}

\begin{exm}
\end{exm}

\begin{dfn}
Если $X= \emptyset$, то $\bigslant{M}{\equiv}$ - свободная группа порождённая $A$
\end{dfn}

\begin{cnsq}
Любая группа порождённая $A$ - гомоморфный образ свободной группы
\end{cnsq}

\begin{dfn}
$\mathcal{G}$ - группа, $S\neq \emptyset$. Действие группы $\mathcal{G}$ на $S$ - это отображение $h:S\times \mathcal{G}\rightarrow S$ и
\begin{enumerate}
\item $h(S,e)=S$
\item $h(h(S,a),b)=h(S,ab)$
\end{enumerate}
Эти два условия по другому:
\begin{enumerate}
\item $Se=S$
\item $(Sa)b=S(ab)$
\end{enumerate}
\end{dfn}

\begin{exm}
$\mathcal{G}$ действует на себя правыми умножениями
\end{exm}

\begin{dfn}
Сопряжение - действие группы $\mathcal{G}$ на себя или множество подмножеств $P(\mathcal{G}):h(S,a)=a^{-1}Sa$
\end{dfn}

\begin{thm}
Сопряжение - действие
\begin{proof}
Проверим условия сопряжения
\begin{enumerate}
\item $e^{-1}Se=eSe=S$
\item $h(h(S,a)b)=h(a^{-1}Sa,b)=b^{-1}a^{-1}Sab=(ab)^{-1}Sab=h(S,ab)$

$a^{-1}Aa=A\subseteq \mathcal{G}$
\end{enumerate}
\end{proof}
\end{thm}

\begin{thm}
Любая подгруппа при сопряжении переходит в подгруппу
\begin{proof}
Пусть $A$ - подгруппа $\mathcal{G}$
\end{proof}
\end{thm}

\begin{thm}
Пусть $A$ - подгруппа, то $A$ неподвижна при всех сопряжениях тогда и только тогда когда $A$ - нормальная подгруппа
\begin{proof}
\begin{itemize}
\item $\Rightarrow$ $a^{-1}Aa=a\Rightarrow aa^{-1}Aa=aA\Rightarrow Aa=aA$
\item $\Leftarrow$ $Aa=aA\Rightarrow a^{-1}Aa=a^{-1}aA\Rightarrow a^{-1}Aa=A$
\end{itemize}
\end{proof}
\end{thm}

\begin{dfn}
$\mathcal{G}$ действует на $S$, $s\in S$. Стабилизатор $s$ - $\stab s=\{a\in \mathcal{G},h(s,a)=s\}$
\end{dfn}

\begin{thm}
$\stab s$ - подгруппа $\mathcal{G}$
\begin{proof}
пусть $b,c \in \stab s$, тогда
\begin{multline*}

\end{multline*}

\end{proof}
\end{thm}



\end{document}