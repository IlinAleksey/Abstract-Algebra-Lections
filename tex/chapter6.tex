\documentclass[../main/document.tex]{subfiles}

\begin{document}
\section{Группы}
\begin{dfn}[Группа]
Группа - моноид, в котором все элементы обратимы
\end{dfn}
\begin{dfn}[Тривиальная группа]
Тривиальная группа - группа, состоящая из одного элемента
\end{dfn}
\begin{thm}
Если $M$ - моноид и $G\subseteq M$ - подмножество обратимых элементов, то $G$ - группа
\begin{proof}

$G\subseteq M$ следовательно $G$ ассоциативна, 
$e$ - обратимый следовательно $G$ имеет нейтральный элемент. 
Надо доказать замкнутость: $x*y\in G$

$x',y'$ - обратные к $x$ и $y$ элементы, тогда
$$(x*y)*(y'*x')=x*(y*y')*x'=x*e*x'=x*x'=e$$
$$(y'*x')*(x'*y')=y'*(x'*x)*y=y*e*y'=y*y'=e$$
$x*y$ обратим $\Rightarrow xy\in G$

если $x\in G$, то $x'*x=x*x'=e$, тогда $x'$  имеет обратный элемент, тогда $x'\in G$. Любой элемент $G$ имеет обратный.

$G$ - группа. Теорема доказана.

\end{proof}
\end{thm}
\begin{thm}[Теорема Гротендика]
Каждый коммутативный моноид, в котором все элементы сократимы можно вложить в группу
\begin{proof}
Пусть $M$ - коммутативный моноид, $G'=M\times M=(a,b)$, где $a,b\in M$, $(a_1,b_1)(a_2,b_2)=(a_1a_2,b_1b_2)$, $(e_1,e_2)$ - нейтральный элемент.

Пусть $(a,b)\equiv (c,d)\Leftrightarrow ad=bc$. Является ли $\equiv$ конгруэнтностью?

\begin{enumerate}
\item $(a,b)\equiv(a,b)$, $ab=ba$
\item $(a,b)\equiv(c,d)$, $ad=bc\Rightarrow cb=da\Rightarrow (c,d)\equiv (a,b)$ 
\item $(a,b)\equiv(c,d)\equiv(u,v)\Rightarrow(a,b)\equiv(u,v)$
\end{enumerate}

Надо доказать:
$$(a_1,b_1)\equiv(a_2,b_2), (c_1,d_1)\equiv(c_2,d_2)\Rightarrow(a_1c_1,b_1d_1)\equiv(a_2c_2,b_2d_2)$$

\begin{multline*}
(a_1,b_1)\equiv(a_2,b_2), (c_1,d_1)\equiv(c_2,d_2)\Rightarrow \\ 
a_1b_2=b_1a_2, c_1d_2=d_1c_2\Rightarrow a_1b_2c_1d_2=b_1a_2d_1c_2\Rightarrow \\
 (a_1c_1)(b_2d_2)=(b_1d_1)(a_2c_2)\Rightarrow \\
 (a_1c_1,b_1d_1)\equiv(a_2c_2,b_2d_2)
\end{multline*}

$(a,b)\equiv (c,d)\Leftrightarrow ad=bc$ - конгруэнтность
\vspace{1em}

Пусть $G = \bigslant{G'}{\equiv}$ надо доказать что $G$ - группа и $M$ вкладывается в $G$

$$ab=ba\Rightarrow abe=ab=ba=bae\Rightarrow (ab,ba)\equiv(e,e)$$
$$\widehat{(a,b)}*\widehat{(b,a)}=\widehat{(ab,ba)}=\widehat{(e,e)}$$
$\Rightarrow$ каждый элемент $G$ имеет обратный $\Rightarrow$ $G$ - группа

Пусть $h:M\rightarrow G$ и $h(a)=\widehat{(a,e)}$, тогда
$$h(ab)=\widehat{(ab,e)}=\widehat{(a,e)}\widehat{(b,e)}=h(a)h(b)$$
$$h(e)=\widehat{(e,e)}$$
$h$ - гомоморфизм

Пусть $h(a)=h(b)$
$$\widehat{(a,e)}=\widehat{(b,e)}\Rightarrow (a,e)\equiv(b,e)\Rightarrow ae=eb \Rightarrow a=b$$
следовательно $h$ - инъекция, следовательно $h$ - вложение

\end{proof}
\end{thm}
\begin{exm}[Пример на теорему Гротендика]
\end{exm}
\begin{thm}\label{group-qualities}
$G$ - группа тогда и только тогда, когда
\begin{enumerate}
\item $(xy)z=x(yz)$
\item $xe=x$
\item $xx^{-1}=e$
\end{enumerate}
\begin{proof}
\begin{enumerate}
\item $\Rightarrow$ по определению группы
\item $\Leftarrow$

$(xy)z=x(yz)\Rightarrow$ $G$ ассоциативна

$xx^{-1}=e\Rightarrow x^{-1}x=e$

Надо доказать: $ex=x$ для любого $x$
\begin{multline}
x^{-1}x=x^{-1}xe=x^{-1}x(x^{-1}x)(x^{-1}x)^{-1}=x^{-1}(xx^{-1})x(x^{-1}x)^{-1}=\\
x^{-1}ex(x^{-1}x)^{-1}=(x^{-1}x)(x^{-1}x)^{-1}=e
\end{multline}
$$ex=(xx^{-1})x=x(x^{-1}x)=xe=x$$
$G$ - группа
\end{enumerate}
\end{proof}
\end{thm}
\begin{cnsq}
Группы образуют многообразие в сигнатуре $(*,e,{ }^{-1})$
\end{cnsq}
\begin{dfn}[Аддитивная группа]
Аддитивная группа - группа со сложением
\end{dfn}
\begin{exm}[Примеры аддитивных групп]
$(\mathbb{Z};+)$
\end{exm}
\begin{dfn}[Мультипликативная группа]
Мультипликативная группа - группа с умножением
\end{dfn}
\begin{exm}[Примеры мультипликативных групп]
$(\mathbb{Q};\cdot)$
\end{exm}
\begin{dfn}[Множество вычетов]
\end{dfn}
\begin{exm}[Пример Множества вычетов]
\end{exm}
\begin{dfn}[Матричная группа]
Матричные группы: носитель группы - $M^*_n(R)$ и $det\neq 0$
\end{dfn}
\begin{exm}[Примеры матричных групп]
\begin{enumerate}
\item $(M^*_n,\cdot,E,{ }^{-1})$ - группа, не коммутативная
\item $det=\pm 1$ - группа
\item $O_n$ - ортогональные, $(O_n,\cdot,E,{ }^{-1})$ - группа
\end{enumerate}
\end{exm}
\begin{dfn}[Группа перестановок]
Группа перестановок - группа перестановок множества $S$ называется группа всех биекций $f:S\rightarrow S$. $(F,\circ,e,{ }^{-1})$
\end{dfn}
\begin{exm}[Пример группы перестановок]
\end{exm}
\begin{dfn}[Симметрическая группа порядка]
Симметрическая группа порядка $n$: $S$ - конечно и состоит из $n$ элементов. $(A,\circ,e,{ }^{-1})$, $A$ - множество автоморфизмов $h:S\rightarrow S$
\end{dfn}
\begin{exm}[Пример симметрической группы] \label{TriangleGroup}
Пример симметрической группы:

\begin{tikzpicture}
\draw (0,0) -- (2,3.46410161514) -- (4,0)-- (0,0);
\draw	(0,0) node[anchor=north east] {2}
		(2,3.46410161514) node[anchor=south] {1}
		(4,0) node[anchor=north west] {3};
\end{tikzpicture}

$A=\{e,r_1,r_2,s_1,s_2,s_3\}$
\begin{itemize}

  \item $e$ - тождественное преобразование
  \item $r_1, r_2$ - поворот на $120^{\circ}$ и $240^{\circ}$ соответственно
  \item $s_1, s_2, s_3$ - оборот вокруг высоты, идущей из первой, второй и третьей вершины соответственно
\end{itemize}

$$\mathbf{D}_3=(A,\circ)$$

\begin{table}[H]
\centering
\caption*{Таблица умножения $\circ$}
\renewcommand*{\arraystretch}{1.4}
\begin{tabular}{c|c|c|c|c|c|c}
  & $e$ & $r_1$ & $r_2$ & $s_1$& $s_2$ & $s_3$  \\ \hline
$e$ & $e$ & $x$ & $e$ & $x$& $e$ & $x$ \\ \hline
$r_1$ & $e$ & $x$ & $e$ & $x$& $e$ & $x$ \\ \hline
$r_2$ & $e$ & $x$ & $e$ & $x$& $e$ & $x$ \\ \hline
$s_1$ & $e$ & $x$ & $e$ & $x$& $e$ & $x$ \\ \hline
$s_2$ & $e$ & $x$ & $e$ & $x$& $e$ & $x$ \\ \hline
$s_3$ & $x$ & $x$& $e$ & $x$& $e$ & $x$ \\ 
\end{tabular}
\end{table}
\end{exm}

\begin{dfn}[Группа кос]
Группа кос - 

\begin{tikzpicture}
\braid a_2 a_1;
\end{tikzpicture}

ещё:

\begin{tikzpicture}
\braid[number of strands=3];
\end{tikzpicture}

потом соображу как длиннее сделать
\end{dfn}
\begin{thm}
Если $G$ - полугруппа, то $G$ является группой тогда и только тогда, когда любое уравнение вида $ax=b$ или $xa=b$, $(a,b\in G)$ имеет в $G$ решение
\begin{proof}
\begin{enumerate}
\item $\Rightarrow$
\begin{align*}
ax&=b & xa&=b \\
a^{-1}ax&=a^{-1}b & xaa^{-1}&=ba^{-1} \\
x&=a^{-1}b & x&=ba^{-1}
\end{align*}
любое уравнение вида $ax=b$ или $xa=b$, $(a,b\in G)$ имеет в $G$ решение
\item $\Leftarrow$ по теореме \ref{group-qualities}
\begin{enumerate}
\item по определению полугруппы
\item $ax=a\Rightarrow x=e$
$ya=b$, имеет решение $y=d$, $da=b$
$$be=dae=da=b\Rightarrow be=b$$
\item для любых $ax=e$ существует решение $x=a^{-1}$ - обратное к $a$
\end{enumerate}
\end{enumerate}
\end{proof}
\end{thm}
\begin{thm}
\begin{enumerate}
\item $(ab)^{-1}=b^{-1}a^{-1}$
\item $(a^{-1})^{-1}=a$
\end{enumerate}
\end{thm}
\begin{dfn}[Абелева группа]
Абелева группа - группа, в которой $xy=yx$
\end{dfn}

\end{document}
