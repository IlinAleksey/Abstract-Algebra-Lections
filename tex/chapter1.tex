\documentclass[../main/document.tex]{subfiles}

\begin{document}
\section{Основные понятия}
\begin{dfn}
\textbf{Сигнатура} - множество имён операций с указанием их местности.

$$(f^{(2)},g^{(3)},h^{(0)}), (+^{(2)},\cdot^{(3)})$$
$h^{(0)}$ - символ константы,
$V$ - имена переменных
\end{dfn}

\begin{dfn}
\textbf{Терм} - выражение, составленное из символов сигнатуры и переменных

\begin{enumerate}
\item $x\in V$, $x$ - терм
\item $c$ - символ константы, с - терм
\item если $t_1,...,t_n$ - термы и $f$ - символ $n$-местной операции, то $f(t_1,...,t_n)$ - терм
\end{enumerate}
\end{dfn}

\begin{exm}
Примеры термов: $-(x),-(0),+(x,y),2+3+a$
\end{exm}

\begin{dfn}
\textbf{Замкнутый терм} - терм, не содержащий переменных
\end{dfn}

\begin{dfn}
\textbf{Универсальная алгебра} - пусть $\Sigma$ - сигнатура, тогда \textit{универсальная алгебра} сигнатуры $\Sigma$ - это пара вида $(A,I)$, где $A$ - произвольное непустое множество, а $I$ - некоторое отображение, которое для всякого $p^{(m)}\in \Sigma$, $I(p^{(m)})$ - $n$-местной операции на множестве
\end{dfn}

\begin{exm}
Пример универсальной алгебры: пусть $\Sigma=(+^{(2)},\cdot^{(2)},-^{(1)},0^{(0)},1^{(0)})$, тогда
\begin{equation*}
\begin{split}
R=(\mathbb{R},I) :& I(+) - \text{сложение}\\
& I(\cdot) - \text{умножение}\\
& I(-) - \text{вычитание}\\
& I(0) - 0\\
& I(1) - 1
\end{split}
\end{equation*}
\end{exm}

\begin{dfn}
$\mathbb{R}$ называется \textbf{основным множеством} или носителем алгебры, а \
$I$ - интерпретацией или интерпретирующей функцией
\end{dfn}

\begin{dfn}
\textbf{Состояние} - функция, приписывающая переменной некоторый элемент носителя $\sigma :V\rightarrow A$
\end{dfn}

\begin{exm}
Пример состояний: $\sigma = \left\{(x,3),(y,-8)\right\}, \sigma(x)=3$
\end{exm}

\begin{dfn}
Значение терма на состоянии - значение того выражения, в котором переменные заменены их значениями

\begin{enumerate}
\item $t$ - переменная, $\sigma(t)$ - по определению состояния
\item $t$ - символ константы, $I(t)=\sigma(t_1)=v_1$
\item если $t_1,...,t_n$ - термы и $\sigma(t_1)=v_1,...,\sigma(t_n)=v_n$ , то $\sigma(t)=I(f)(v_1,...,v_n)$
\end{enumerate}
\end{dfn}
%% my chapter 1 content
%%
%% more of my chapter 1 content
\end{document}
